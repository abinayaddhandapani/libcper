\documentclass{report}
\usepackage{hyperref}
\usepackage{adjustbox}
\usepackage{placeins}

% Metadata.
\title{CPER-JSON Specification}
\author{\parbox{\linewidth}{\centering%
Lawrence Tang\endgraf
Lawrence.Tang@arm.com\endgraf\medskip}}
\date{\parbox{\linewidth}{\centering%
Revision v0.0.1 (\today)\endgraf
First revision released [DATE].}}

% Commands.
\newcommand*{\thead}[1]{\multicolumn{1}{|c|}{\bfseries #1}}
\newcommand*{\jsontable}[1]{
    \begin{table}[!ht]
    \label{#1}
    \centering
    \begin{adjustbox}{center}
    \begin{tabular}{|l|c|p{8cm}|}
    \hline
    \thead{Field Name} & \thead{Type} & \thead{Description} \\
    \hline
}
\newcommand*{\jsontableend}[1]{
    \hline
    \end{tabular}
    \end{adjustbox}
    \caption{#1}
    \label{table:#1}
    \end{table}
    \FloatBarrier
}
    
\begin{document}
\maketitle
\tableofcontents
\listoftables

% Introductory section.
\chapter{Preface}
\section{Introduction \& Summary}
This document lays out a structure for representing UEFI CPER records, as described in UEFI Appendix N
\footnote{Version referenced is \href{https://uefi.org/sites/default/files/resources/UEFI_Spec_2_9_2021_03_18.pdf}{UEFI Specification 2021/03/18}.},
 in a human-readable JSON format, intended to be interoperable with standard CPER binary.
\\\\
The C library released with this specification allows for the conversion between UEFI CPER records, an intermediate format, and the JSON structures
defined in this document.

% Specification section.
\chapter{Main Structure Specification}
\section{Parent Structure}
\label{section:parentstructure}
This structure contains descriptions of the CPER log header, as well as the section descriptors and
section structures themselves within arrays. This is the structure returned by \texttt{cper\_to\_ir(FILE* cper\_file)} as JSON IR.

% Parent structure table.
\jsontable{table:parentstructure}
header & object & A CPER header structure as defined in Section \ref{section:headerstructure}. \\
\hline
sectionDescriptors & array & An array of section descriptor objects as defined in Section \ref{section:sectiondescriptorstructure}. \\
\hline
sections & array & An array of section objects as defined in Chapter \ref{chapter:sectionchapter}. These sections are at the same index as their corresponding section descriptor within the \texttt{sectionDescriptors} array.\\
\jsontableend{Parent structure field table.}

% Header structure.
\section{Header Structure}
\label{section:headerstructure}
This structure describes the JSON format of the standard CPER header as defined in section N.2.1 of the
UEFI specification.

% Header structure table.
\jsontable{table:headerstructure}
revision & object & A CPER revision object structure as defined in Subsection \ref{subsection:revisionstructure}. \\
\hline
sectionCount & int & The number of sections that are described by the CPER record.\\
\hline
severity & object & An error severity structure as described in \ref{subsection:headererrorseveritystructure}.\\
\hline
validationBits & object & A CPER header validation bitfield as described in Subsection \ref{subsection:headervalidbitfieldstructure}.\\
\hline
recordLength & uint64 & The total length of the binary CPER record, including the header, in bytes.\\
\hline
timestamp & string (\textbf{optional}) & The attached record timestamp, if the validity field is set. Formatted identically to \texttt{Date.toJson()} (ISO 8601), minus the trailing timezone letter. Timezone is local to the machine creating the record.\\
\hline
timestampIsPrecise & boolean (\textbf{optional}) & If a timestamp is attached, indicates whether the provided timestamp is precise.\\
\hline
platformID & string (\textbf{optional}) & If validation bit is set, uniquely identifying GUID of the platform. Platform SMBIOS UUID should be used to populate this field.\\
\hline
partitionID & string (\textbf{optional}) & If validation bit is set, GUID identifying the partition on which the error occurred.\\
\hline
creatorID & string & A GUID identifying the creator of the error record. May be overwritten by subsequent owners of the record.\\
\hline
notificationType & object & A CPER notification type structure as described in Subsection \ref{subsection:notificationtypestructure}.\\
\hline
recordID & uint64 & A unique value which, when combined with the \texttt{creatorID} field, uniquely identifies this error record on a given system.\\
\hline
flags & object & A CPER header flags structure, as defined in Subsection \ref{subsection:headerflagsstructure}.\\
\hline
persistenceInfo & uint64 & Produced and consumed by the creator of the error record identified by \texttt{creatorID}. Format undefined.\\
\jsontableend{Header structure field table.}

% Header error severity.
\subsection{Header Error Severity Structure}
\label{subsection:headererrorseveritystructure}
This structure describes the error severity of a single CPER record.
\jsontable{table:headererrorseveritystructure}
name & string & The human readable name of this error severity, if known. \\
\hline
code & uint64 & The integer value of this error severity. \\
\jsontableend{Header error severity structure field table.}

% Header validation bitfield.
\subsection{Header Validation Bitfield Structure}
\label{subsection:headervalidbitfieldstructure}
This structure describes a bitfield for validating the fields of the header of a single CPER record.
\jsontable{table:headervalidbitfieldstructure}
platformIDValid & boolean & Whether the "platformID" field in the header structure (\ref{section:headerstructure}) is valid. \\
\hline
timestampValid & boolean & Whether the "timestamp" field in the header structure (\ref{section:headerstructure}) is valid. \\
\hline
partitionIDValid & boolean & Whether the "partitionID" field in the header structure (\ref{section:headerstructure}) is valid.\\
\jsontableend{Header validation bitfield structure field table.}

% Header notification type.
\subsection{Notification Type Structure}
\label{subsection:notificationtypestructure}
This structure describes the notification type of a single CPER record.
\jsontable{table:notificationtypestructure}
guid & string & The GUID of this notification type. Assigned GUIDs for types of CPER records are defined in UEFI Specification section N.2.1.1.\\
\hline
type & string & A human readable name, if available, of the notification type for the given GUID.\\
\jsontableend{Notification type structure field table.}

% Header flags.
\subsection{Header Flags Structure}
\label{subsection:headerflagsstructure}
This structure describes the enabled flag on a given CPER record header.
\jsontable{table:headerflagsstructure}
name & string & A human readable name, if available, of this flag.\\
\hline
value & uint64 & The integer value of this flag.\\
\jsontableend{Header flags structure field table.}

%Section descriptor structure.
\section{Section Descriptor Structure}
\label{section:sectiondescriptorstructure}
This section describes the JSON format of a single CPER record section descriptor as defined by section N.2.2 of the UEFI specification. An array of these structures is contained within the parent structure as defined in Section \ref{section:parentstructure}.

%Section descriptor structure table.
\jsontable{table:sectiondescriptorstructure}
sectionOffset & uint64 & The offset (in bytes) of the section body this section descriptor describes from the base of the record header.\\
\hline
sectionLength & uint64 & The length (in bytes) of the section body.\\
\hline
revision & object & A CPER revision structure as defined in Subsection \ref{subsection:revisionstructure}.\\
\hline
validationBits.fruIDValid & boolean & Whether the "fruID" field on this section descriptor contains valid data.\\
validationBits.fruStringValid & boolean & Whether the "fruString" field on this section descriptor contains valid data.\\
\hline
flags & object & A CPER section descriptor flags structure as described in Subsection \ref{subsection:sectiondescriptorflagsstructure}.\\
\hline
sectionType.data & string & GUID data for the type of section body.\\
sectionType.type & string & The human readable name, if possible, for the type of section body. GUIDs for types of sectoin body are defined in UEFI specification section N.2.2 Table N-5 and section N.2.4.\\
\hline
fruID & string (\textbf{optional}) & If validation field set, the FRU ID of the section reporting the error.\\
\hline
severity.code & uint64 & The integer value of the severity of the described section.\\
severity.name & string & If available, the human readable name for the severity of the described section.\\
\hline
fruText & string (\textbf{optional}) & If validation field set, ASCII string identifying the FRU hardware.\\
\jsontableend{Section descriptor structure field table.}

% Section descriptor flags.
\subsection{Section Descriptor Flags Structure}
\label{subsection:sectiondescriptorflagsstructure}
This structure describes the enabled flags on a given CPER section descriptor.
\jsontable{table:sectiondescriptorflagsstructure}
primary & boolean & If true, indicates the section body should be associated with the error condition.\\
\hline
containmentWarning & boolean & If true, the error was not contained within the processor or memory heirarchy, and may have propagated elsewhere.\\
\hline
reset & boolean & If true, indicates the component has been reset and must be re-initialised or re-enabled by the operating system.\\
\hline
errorThresholdExceeded & boolean & If true, indicates the operating system may choose to discontinue use of this resource.\\
\hline
resourceNotAccessible & boolean & If true, the resource could not be queried for error information due to conflicts with other system software or resources. Some fields of the section will be invalid.\\
\hline
latentError & boolean & If true, indicates that action has been taken to ensure error containment, but the error has not been fully corrected. System software may choose to take further action before the data is consumed.\\
\hline
propagated & boolean & If true, indicates that the error has been propagated due to hardware poisoning.\\
\hline
overflow & boolean & If true, overflow of data structures used to manage errors has been detected. Some error records may be lost.\\
\jsontableend{Section descriptor flags structure field table.}

% Generic CPER structures.
\section{Generic CPER Structures}
This section describes generic CPER structures that are re-used throughout the specification.

% Revision.
\subsection{Revision Structure}
\label{subsection:revisionstructure}
This structure describes the revision of a single CPER record or sub-structure.
\jsontable{table:revisionstructure}
major & int & The major version number. An increase in this revision indicates the changes are not backward compatible. \\
\hline
minor & int & The minor version number. Incremented on additions of new GUID types, errata fixes, or clarifications. Backwards compatible with the same major version number. \\
\jsontableend{CPER revision structure field table.}

% Generic Error Status
\subsection{Generic Error Status Structure}
\label{subsection:genericerrorstatusstructure}
This structure describes the a generic CPER error status, that can be used by any CPER section.
\jsontable{table:genericerrorstatusstructure}
errorType.value & uint64 & The raw value of the error type.\\
errorType.name & string & The human readable name, if available, of the error type.\\
errorType.description & string & A human readable description, if available, of the error type.\\
\hline
addressSignal & boolean & Whether the error was detected on the address signals/address portion of the transaction.\\
\hline
controlSignal & boolean & Whether the error was detected on the control signals/control portion of the transaction.\\
\hline
dataSignal & boolean & Whether the error was detected on the data signals/data portion of the transaction.\\
\hline
detectedByResponder & boolean & Whether the error was detected by the responder of the transaction.\\
\hline
detectedByRequester & boolean & Whether the error was detected by the requester of the transaction.\\
\hline
firstError & boolean & If multiple errors are logged for a single type of section, this indicates this is the first error in a chronological sequence. This field is optional to set.\\
\hline
overflowDroppedLogs & boolean & Additional errors occurred and were not logged due to lack of resources.\\
\jsontableend{Generic Error Status structure field table.}

%Sections.
\chapter{Section Specification}
\label{chapter:sectionchapter}
This chapter defines section body formats for all of the sections defined within UEFI Specification section N.2.4.

% Generic processor error section.
\section{Generic Processor Error Section}
\label{section:genericprocessorerrorsection}
This section describes the JSON format for a single Generic Processor Error Section from a CPER record. The GUID used for Generic Processor Error Sections is \texttt{\{x9876CCAD, 0x47B4, 0x4bdb, \{0xB6, 0x5E, 0x16, 0xF1, 0x93, 0xC4, 0xF3, 0xDB\}\}}.
\jsontable{table:genericprocessorerrorsection}
validationBits & object & A Generic Processor Error Validation Structure, as described in Subsection \ref{subsection:genericprocessorvalidationstructure}.\\
\hline
processorType.name & string & If available, the human readable name of the processor type.\\
processorType.value & uint64 & The integer value of the processor type.\\
\hline
processorISA.name & string & If available, the human readable name of the processor ISA.\\
processorISA.value & uint64 & The integer value corresponding to the processor ISA.\\
\hline
errorType.name & string & If available, the human readable name of the type of processor error this section describes.\\
errorType.value & uint64 & The integer value corresponding to the processor error type.\\
\hline
operation.name & string & If available, the human readable name of the operation.\\
operation.value & uint64 & The integer value corresponding to the operation.\\
\hline
flags & object & Flag information for the Generic Processor Error as described in Subsection \ref{subsection:genericprocessorflagsstructure}.\\
\hline
level & int & The level of the structure at which the error occurred.\\
\hline
cpuVersionInfo & uint64 & The CPU version information as reported by CPUID with EAX=1. On ARM, this is MIDR\_EL1.\\
\hline
cpuBrandString & string & The ASCII brand string of the CPU. This field is optional on ARM.\\
\hline
processorID & uint64 & The unique identifier of the logical processor. On ARM, this is MPIDR\_EL1.\\
\hline
targetAddress & uint64 & The target address associated with the error.\\
\hline
requestorID & uint64 & ID of the requestor associated with the error.\\
\hline
responderID & uint64 & ID of the responder associated with the error.\\
\hline
instructionIP & uint64 & Identifies the instruction pointer at the point of error.\\
\jsontableend{Generic Processor Error structure field table.}

% Generic processor error validation structure.
\subsection{Generic Processor Error Validation Structure}
\label{subsection:genericprocessorvalidationstructure}
This structure describes the valdation bits structure of a General Processor Error CPER section.
\jsontable{table:genericprocessorvalidationstructure}
processorTypeValid & boolean & Whether the "processorType" field of the Generic Processor Error section (\ref{section:genericprocessorerrorsection}) is valid.\\
\hline
processorISAValid & boolean & Whether the "processorISA" field of the Generic Processor Error section (\ref{section:genericprocessorerrorsection}) is valid.\\
\hline
errorTypeValid & boolean & Whether the "errorType" field of the Generic Processor Error section (\ref{section:genericprocessorerrorsection}) is valid.\\
\hline
operationValid & boolean & Whether the "operation" field of the Generic Processor Error section (\ref{section:genericprocessorerrorsection}) is valid.\\
\hline
flagsValid & boolean & Whether the "flags" field of the Generic Processor Error section (\ref{section:genericprocessorerrorsection}) is valid.\\
\hline
levelValid & boolean & Whether the "levelValid" field of the Generic Processor Error section (\ref{section:genericprocessorerrorsection}) is valid.\\
\hline
cpuVersionValid & boolean & Whether the "cpuVersion" field of the Generic Processor Error section (\ref{section:genericprocessorerrorsection}) is valid.\\
\hline
cpuBrandInfoValid & boolean & Whether the "cpuBrandInfo" field of the Generic Processor Error section (\ref{section:genericprocessorerrorsection}) is valid.\\
\hline
cpuIDValid & boolean & Whether the "cpuID" field of the Generic Processor Error section (\ref{section:genericprocessorerrorsection}) is valid.\\
\hline
targetAddressValid & boolean & Whether the "targetAddress" field of the Generic Processor Error section (\ref{section:genericprocessorerrorsection}) is valid.\\
\hline
requesterIDValid & boolean & Whether the "requesterID" field of the Generic Processor Error section (\ref{section:genericprocessorerrorsection}) is valid.\\
\hline
responderIDValid & boolean & Whether the "responderID" field of the Generic Processor Error section (\ref{section:genericprocessorerrorsection}) is valid.\\
\hline
instructionIPValid & boolean & Whether the "instructionIP" field of the Generic Processor Error section (\ref{section:genericprocessorerrorsection}) is valid.\\
\jsontableend{Generic Processor Error validation structure field table.}

% Generic processor error flags structure.
\subsection{Generic Processor Error Flags Structure}
\label{subsection:genericprocessorflagsstructure}
This structure describes the flags structure of a General Processor Error CPER section.
\jsontable{table:genericprocessorflagsstructure}
restartable & boolean & Whether program execution can be restarted reliably after the error.\\
\hline
preciseIP & boolean & Whether the instruction IP captured is directly associated with the error.\\
\hline
overflow & boolean & Whether a machine check overflow occurred (multiple errors occurred at once).\\
\hline
corrected & boolean & Whether the error was corrected by hardware/firmware.\\
\jsontableend{Generic Processor Error flags structure field table.}

% IA32/x64 error section.
\section{IA32/x64 Processor Error Section}
\label{section:ia32x64errorsection}
This section describes the JSON format for a single IA32/x64 Error Section from a CPER record. The GUID used for IA32/x64 Processor Error Sections is \texttt{\{0xDC3EA0B0, 0xA144, 0x4797, \{0xB9, 0x5B, 0x53, 0xFA, 0x24, 0x2B, 0x6E, 0x1D\}\}}.
\jsontable{table:genericprocessorerrorsection}
validationBits & object & IA32/x64 Processor Error Validation Structure as described in Subsection \ref{subsection:ia32x64processorflagsstructure}.\\
\hline
localAPICID & uint64 & The APIC ID of the processor.\\
\hline
cpuidInfo & object & IA32/x64 CPUINFO Structure as defined in Subsection \ref{subsection:ia32x64cpuinfostructure}.\\
\hline
processorErrorInfo & array & Array of IA32/x64 Processor Error Info Structures as described in Subsection \ref{subsection:ia32x64processorerrorinfostructure}.\\
\hline
processorContextInfo & array & Array of IA32/x64 Processor Context Info Structures as described in Subsection \ref{subsection:ia32x64processorcontextinfostructure}.\\
\jsontableend{IA32/x64 Processor Error structure field table.}

% IA32/x64 validation bitfield structure.
\subsection{IA32/x64 Processor Error Validation Structure}
\label{subsection:ia32x64processorflagsstructure}
This structure describes the validation bitfield structure of an IA32/x64 Error CPER section.
\jsontable{table:ia32x64processorflagsstructure}
localAPICIDValid & boolean & Whether the "localAPICID" field of the IA32/x64 Error section (\ref{section:ia32x64errorsection}) is valid.\\
\hline
cpuIDInfoValid & boolean & Whether the "cpuIDInfo" field of the IA32/x64 Error section (\ref{section:ia32x64errorsection}) is valid.\\ 
\hline
processorErrorInfoNum & int & The number of IA32/x64 Processor Error Info Structures (\ref{subsection:ia32x64processorerrorinfostructure}) that are included with this error section.\\
\hline
processorContextInfoNum & int & The number of IA32/x64 Processor Context Info Structures (\ref{subsection:ia32x64processorcontextinfostructure}) that are included with this error section.\\
\jsontableend{IA32/x64 Processor Error validation structure field table.}

% IA32/x64 CPUINFO structure.
\subsection{IA32/x64 CPUINFO Structure}
\label{subsection:ia32x64cpuinfostructure}
This structure describes the CPUINFO structure of an IA32/x64 Error CPER section.
\jsontable{table:ia32x64cpuinfostructure}
eax & uint64 & Value of the EAX register resulting from a call to CPUID with EAX=1.\\
\hline
ebx & uint64 & Value of the EBX register resulting from a call to CPUID with EAX=1.\\
\hline
ecx & uint64 & Value of the ECX register resulting from a call to CPUID with EAX=1.\\
\hline
edx & uint64 & Value of the EDX register resulting from a call to CPUID with EAX=1.\\
\jsontableend{IA32/x64 CPUINFO structure field table.}

% IA32/x64 Processor Error Info structure.
\subsection{IA32/x64 Processor Error Info Structure}
\label{subsection:ia32x64processorerrorinfostructure}
This structure describes a single IA32/x64 Processor Error Info sub-section, which is part of the larger IA32/x64 record (\ref{section:ia32x64errorsection}).
\jsontable{table:ia32x64processorerrorinfostructure}
type & string & A GUID indicating the type of processor error defined in this structure. See \texttt{edk/Cper.h} in the library repository for the defined GUID values.\\
\hline
validationBits & object & An IA32/x64 Processor Error Info Validation structure, as defined in Subsection \ref{subsection:ia32x64processorerrorinfovalidationstructure}.\\
\hline
checkInfo & object & Check information structure for this error. One of the structures defined in Subsections \ref{subsection:ia32x64processorerrorcheckinfocachetlbstructure}, \ref{subsection:ia32x64processorerrorcheckinfobusstructure}, or \ref{subsection:ia32x64processorerrorcheckinfomscheckstructure}. Which section is placed here is dependent on the \texttt{type} field.\\
\hline
targetAddressID & uint64 & Identifies the target address associated with the error.\\
\hline
requestorID & uint64 & Identifies the requestor associated with the error.\\
\hline
responderID & uint64 & Identifies the responder associated with the error.\\
\hline
instructionPointer & uint64 & Identifies the instruction executing when the error occurred.\\ 
\jsontableend{IA32/x64 Processor Error Info structure field table.}

% IA32/x64 Processor Error Info Validation structure.
\subsection{IA32/x64 Processor Error Info Validation Structure}
\label{subsection:ia32x64processorerrorinfovalidationstructure}
This structure describes a single IA32/x64 Processor Error Info structure's valid fields, as a set of boolean values.
\jsontable{table:ia32x64processorerrorinfovalidationstructure}
checkInfoValid & boolean & Whether the "checkInfo" field in the Processor Error Info structure (\ref{subsection:ia32x64processorerrorinfostructure}) is valid.\\
\hline
targetAddressIDValid & boolean & Whether the "targetAddressID" field in the Processor Error Info structure (\ref{subsection:ia32x64processorerrorinfostructure}) is valid.\\
\hline
requestorIDValid & boolean & Whether the "requestorID" field in the Processor Error Info structure (\ref{subsection:ia32x64processorerrorinfostructure}) is valid.\\
\hline
responderIDValid & boolean & Whether the "responderID" field in the Processor Error Info structure (\ref{subsection:ia32x64processorerrorinfostructure}) is valid.\\
\hline
instructionPointerValid & boolean & Whether the "instructionPointer" field in the Processor Error Info structure (\ref{subsection:ia32x64processorerrorinfostructure}) is valid.\\
\hline
\jsontableend{IA32/x64 Processor Error Info validation structure field table.}

% IA32/x64 Processor Error Check Info (Cache/TLB Error)
\subsection{IA32/x64 Processor Error Check Info (Cache/TLB Error) Structure}
\label{subsection:ia32x64processorerrorcheckinfocachetlbstructure}
This structure describes check info for an IA32/x64 Processor Error Info structure (\ref{subsection:ia32x64processorerrorinfostructure}) stemming from a cache or TLB error.
The GUIDs for cache and TLB error check info structures can be found in the library repository's \texttt{edk/Cper.h}.
\jsontable{table:ia32x64processorerrorcheckinfocachetlbstructure}
validationBits & object & An IA32/x64 Processor Error Check Info (Cache/TLB/Bus) Validation structure, as defined in Subsection \ref{subsection:ia32x64processorerrorcheckinfovalidationstructure}.\\
\hline
transactionType.value & uint64 & The raw value of the type of cache/TLB error that occurred.\\
transactionType.name & string & The human readable name, if available, of the type of cache/TLB error that occurred.\\
\hline
operation.value & uint64 & The raw value of the type of cache/TLB operation that caused the error.\\
operation.name & string & The human readable name, if available, of the type of cache/TLB operation that caused the error.\\
\hline
level & uint64 & The cache/TLB level at which the error occurred.\\
\hline
processorContextCorrupt & boolean & Whether the processor context might have been corrupted.\\
\hline
uncorrected & boolean & Whether the error remained uncorrected.\\
\hline
preciseIP & boolean & Whether the instruction pointed pushed onto the stack is directly associated with the error.\\
\hline
restartableIP & boolean & Whether program execution can be restarted reliably at the instruction pointer pushed onto the stack.\\
\hline
overflow & boolean & Whether an error overflow occurred (multiple errors within a short timeframe may cause this, can indicate loss of data).\\
\jsontableend{IA32/x64 Processor Error Check Info (Cache/TLB Error) structure field table.}

% IA32/x64 Processor Error Check Info (Bus Error)
\subsection{IA32/x64 Processor Error Check Info (Bus Error) Structure}
\label{subsection:ia32x64processorerrorcheckinfobusstructure}
This structure describes check info for an IA32/x64 Processor Error Info structure (\ref{subsection:ia32x64processorerrorinfostructure}) stemming from a bus error.
The GUID for bus error check info structures can be found in the library repository's \texttt{edk/Cper.h}.
\jsontable{table:ia32x64processorerrorcheckinfobusstructure}
validationBits & object & An IA32/x64 Processor Error Check Info (Cache/TLB/Bus) Validation structure, as defined in Subsection \ref{subsection:ia32x64processorerrorcheckinfovalidationstructure}.\\
\hline
transactionType.value & uint64 & The raw value of the type of bus error that occurred.\\
transactionType.name & string & The human readable name, if available, of the type of bus error that occurred.\\
\hline
operation.value & uint64 & The raw value of the type of bus operation that caused the error.\\
operation.name & string & The human readable name, if available, of the type of bus operation that caused the error.\\
\hline
level & uint64 & The bus heirarchy level at which the error occurred.\\
\hline
processorContextCorrupt & boolean & Whether the processor context might have been corrupted.\\
\hline
uncorrected & boolean & Whether the error remained uncorrected.\\
\hline
preciseIP & boolean & Whether the instruction pointed pushed onto the stack is directly associated with the error.\\
\hline
restartableIP & boolean & Whether program execution can be restarted reliably at the instruction pointer pushed onto the stack.\\
\hline
overflow & boolean & Whether an error overflow occurred (multiple errors within a short timeframe may cause this, can indicate loss of data).\\
\hline
participationType.value & uint64 & The raw value of the type of participation.\\
participationType.name & string & The human readable name, if available, of the type of participation.\\
\hline
timedOut & boolean & Whether the request timed out.\\
\hline
addressSpace.value & uint64 & The raw value of the address space the error was in.\\
addressSpace.name  & string & The human readable name, if available, of the address space the error was in.\\
\jsontableend{IA32/x64 Processor Error Check Info (Bus Error) structure field table.}

% IA32/x64 Processor Error Check Info (MS Check Error)
\subsection{IA32/x64 Processor Error Check Info (MS Check Error) Structure}
\label{subsection:ia32x64processorerrorcheckinfomscheckstructure}
This structure describes check info for an IA32/x64 Processor Error Info structure (\ref{subsection:ia32x64processorerrorinfostructure}) stemming from an MS check error.
The GUID for MS check check info structures can be found in the library repository's \texttt{edk/Cper.h}.
\jsontable{table:ia32x64processorerrorcheckinfomscheckstructure}
validationBits & object & An IA32/x64 Processor Error Check Info (MS Check) Validation structure, as defined in Subsection \ref{subsection:ia32x64processorerrorcheckinfomscheckvalidationstructure}.\\
\hline
errorType.value & uint64 & The raw value of the type of operation that caused the error.\\
errorType.name & string & The human readable name, if available, of the type of operation that caused the error.\\
\hline
processorContextCorrupt & boolean & Whether the processor context might have been corrupted.\\
\hline
uncorrected & boolean & Whether the error remained uncorrected.\\
\hline
preciseIP & boolean & Whether the instruction pointed pushed onto the stack is directly associated with the error.\\
\hline
restartableIP & boolean & Whether program execution can be restarted reliably at the instruction pointer pushed onto the stack.\\
\hline
overflow & boolean & Whether an error overflow occurred (multiple errors within a short timeframe may cause this, can indicate loss of data).\\
\jsontableend{IA32/x64 Processor Error Check Info (MS Check Error) structure field table.}

% IA32/x64 Processor Error Check Info Validation structure.
\subsection{IA32/x64 Processor Error Check Info (Cache/TLB/Bus) Validation Structure}
\label{subsection:ia32x64processorerrorcheckinfovalidationstructure}
This structure describes a single IA32/x64 Processor Error Check Info structure's valid fields for cache, TLB and bus errors, as a set of boolean values.
\jsontable{table:ia32x64processorerrorcheckinfovalidationstructure}
transactionTypeValid & boolean & Whether the "transactionType" field in a Processor Error Check Info structure is valid.\\
\hline
operationValid & boolean & Whether the "operation" field in a Processor Error Check Info structure is valid.\\
\hline
levelValid & boolean & Whether the "level" field in a Processor Error Check Info structure is valid.\\
\hline
processorContextCorruptValid & boolean & Whether the "processorContextCorrupt" field in a Processor Error Check Info structure is valid.\\
\hline
uncorrectedValid & boolean & Whether the "uncorrected" field in a Processor Error Check Info structure is valid.\\
\hline
preciseIPValid & boolean & Whether the "preciseIP" field in a Processor Error Check Info structure is valid.\\
\hline
restartableIPValid & boolean & Whether the "restartableIP" field in a Processor Error Check Info structure is valid.\\
\hline
overflowValid & boolean & Whether the "overflow" field in a Processor Error Check Info structure is valid.\\
\hline
participationTypeValid & boolean (\textbf{optional}) & Whether the "participationType" field in the Processor Error Check Info (Bus Error) structure (\ref{subsection:ia32x64processorerrorcheckinfobusstructure}) is valid. \textbf{This field is only present on bus related check info structures.}\\
\hline
timedOutValid & boolean (\textbf{optional}) & Whether the "timeOut" field in the Processor Error Check Info (Bus Error) structure (\ref{subsection:ia32x64processorerrorcheckinfobusstructure}) is valid. \textbf{This field is only present on bus related check info structures.}\\
\jsontableend{IA32/x64 Processor Error Check Info (Cache/TLB/Bus) validation structure field table.}

% IA32/x64 Processor Error Check Info (MS Check) Validation structure.
\subsection{IA32/x64 Processor Error Check Info (MS Check) Validation Structure}
\label{subsection:ia32x64processorerrorcheckinfomscheckvalidationstructure}
This structure describes a single IA32/x64 Processor Error Check Info structure's valid fields for MS check errors, as a set of boolean values.
\jsontable{table:ia32x64processorerrorcheckinfomscheckvalidationstructure}
errorTypeValid & boolean & Whether the "transactionType" field in a Processor Error Check Info (MS Check) (\ref{subsection:ia32x64processorerrorcheckinfomscheckstructure}) structure is valid.\\
\hline
processorContextCorruptValid & boolean & Whether the "processorContextCorrupt" field in a Processor Error Check Info (MS Check) (\ref{subsection:ia32x64processorerrorcheckinfomscheckstructure}) structure is valid.\\
\hline
uncorrectedValid & boolean & Whether the "uncorrected" field in a Processor Error Check Info (MS Check) (\ref{subsection:ia32x64processorerrorcheckinfomscheckstructure}) structure is valid.\\
\hline
preciseIPValid & boolean & Whether the "preciseIP" field in a Processor Error Check Info (MS Check) (\ref{subsection:ia32x64processorerrorcheckinfomscheckstructure}) structure is valid.\\
\hline
restartableIPValid & boolean & Whether the "restartableIP" field in a Processor Error Check Info (MS Check) (\ref{subsection:ia32x64processorerrorcheckinfomscheckstructure}) structure is valid.\\
\hline
overflowValid & boolean & Whether the "overflow" field in a Processor Error Check Info (MS Check) (\ref{subsection:ia32x64processorerrorcheckinfomscheckstructure}) structure is valid.\\
\jsontableend{IA32/x64 Processor Error Check Info (MS Check) validation structure field table.}

% IA32/x64 Processor Context Info structure.
\subsection{IA32/x64 Processor Context Info Structure}
\label{subsection:ia32x64processorcontextinfostructure}
This structure describes a single IA32/x64 Processor Context Info sub-section, which is part of the larger IA32/x64 record (\ref{section:ia32x64errorsection}).
\jsontable{table:ia32x64processorcontextinfostructure}
registerContextType.value & uint64 & The raw value of the type of processor context state being reported.\\
registerContextType.name & string & The human readable name, if available, of the type of processor context state being reported.\\
\hline
registerArraySize & uint64 & The total size of the array for the data type being reported, in bytes.\\
\hline
msrAddress & uint64 & The starting MSR address. Valid when the \texttt{registerContextType.value} field is "1" (MSR Registers).\\
\hline
mmRegisterAddress & uint64 & The starting memory address for when the \texttt{registerContextType.value} field is "7" (Memory Mapped Registers).\\
\hline
registerArray & object & Register data, formatted as object fields. If the \texttt{registerContextType.value} field has the value "2" or "3", this takes the structure of Subsections \ref{subsection:ia32x64ia32registerstatestructure} and \ref{subsection:ia32x64x64registerstatestructure} respectively. If the value is any other, it takes the form of the structure defined in Subsection \ref{subsection:ia32x64unformattedregisterstatestructure}.\\
\jsontableend{IA32/x64 Processor Context Info structure field table.}

% IA32/x64 IA32 Register State structure
\subsection{IA32/x64 IA32 Register State Structure}
\label{subsection:ia32x64ia32registerstatestructure}
This structure describes a single IA32/x64 IA32 register state, which is contained in IA32/x64 Processor Context Info structures (\ref{subsection:ia32x64processorcontextinfostructure}) when \texttt{registerContextType.value} has the value "2".
\jsontable{table:ia32x64ia32registerstatestructure}
eax & uint64 & The EAX register. Real maximum is \texttt{UINT32}, null extended to \texttt{UINT64}.\\
\hline
ebx & uint64 & The EBX register. Real maximum is \texttt{UINT32}, null extended to \texttt{UINT64}.\\
\hline
ecx & uint64 & The ECX register. Real maximum is \texttt{UINT32}, null extended to \texttt{UINT64}.\\
\hline
edx & uint64 & The EDX register. Real maximum is \texttt{UINT32}, null extended to \texttt{UINT64}.\\
\hline
esi & uint64 & The ESI register. Real maximum is \texttt{UINT32}, null extended to \texttt{UINT64}.\\
\hline
edi & uint64 & The EDI register. Real maximum is \texttt{UINT32}, null extended to \texttt{UINT64}.\\
\hline
ebp & uint64 & The EBP register. Real maximum is \texttt{UINT32}, null extended to \texttt{UINT64}.\\
\hline
esp & uint64 & The ESP register. Real maximum is \texttt{UINT32}, null extended to \texttt{UINT64}.\\
\hline
cs & uint64 & The CS register. Real maximum is \texttt{UINT32}, null extended to \texttt{UINT64}.\\
\hline
ds & uint64 & The DS register. Real maximum is \texttt{UINT32}, null extended to \texttt{UINT64}.\\
\hline
ss & uint64 & The SS register. Real maximum is \texttt{UINT32}, null extended to \texttt{UINT64}.\\
\hline
es & uint64 & The ES register. Real maximum is \texttt{UINT32}, null extended to \texttt{UINT64}.\\
\hline
fs & uint64 & The FS register. Real maximum is \texttt{UINT32}, null extended to \texttt{UINT64}.\\
\hline
gs & uint64 & The GS register. Real maximum is \texttt{UINT32}, null extended to \texttt{UINT64}.\\
\hline
eflags & uint64 & The EFLAGS register. Real maximum is \texttt{UINT32}, null extended to \texttt{UINT64}.\\
\hline
eip & uint64 & The EIP register. Real maximum is \texttt{UINT32}, null extended to \texttt{UINT64}.\\
\hline
cr0 & uint64 & The CR0 register. Real maximum is \texttt{UINT32}, null extended to \texttt{UINT64}.\\
\hline
cr1 & uint64 & The CR1 register. Real maximum is \texttt{UINT32}, null extended to \texttt{UINT64}.\\
\hline
cr2 & uint64 & The CR2 register. Real maximum is \texttt{UINT32}, null extended to \texttt{UINT64}.\\
\hline
cr3 & uint64 & The CR3 register. Real maximum is \texttt{UINT32}, null extended to \texttt{UINT64}.\\
\hline
cr4 & uint64 & The CR4 register. Real maximum is \texttt{UINT32}, null extended to \texttt{UINT64}.\\
\hline
gdtr & uint64 & The GDTR register.\\
\hline
idtr & uint64 & The IDTR register.\\
\hline
ldtr & uint64 & The LDTR register.\\
\hline
tr & uint64 & The TR register. Real maximum is \texttt{UINT32}, null extended to \texttt{UINT64}.\\
\jsontableend{IA32/x64 IA32 Register State structure field table.}

% IA32/x64 x64 Register State structure
\subsection{IA32/x64 x64 Register State Structure}
\label{subsection:ia32x64x64registerstatestructure}
This structure describes a single IA32/x64 x64 register state, which is contained in IA32/x64 Processor Context Info structures (\ref{subsection:ia32x64processorcontextinfostructure}) when \texttt{registerContextType.value} has the value "3".
\jsontable{table:ia32x64x64registerstatestructure}
rax & uint64 & The RAX register.\\
\hline
rbx & uint64 & The RBX register.\\
\hline
rcx & uint64 & The RCX register.\\
\hline
rdx & uint64 & The RDX register.\\
\hline
rsi & uint64 & The RSI register.\\
\hline
rdi & uint64 & The RDI register.\\
\hline
rbp & uint64 & The RBP register.\\
\hline
rsp & uint64 & The RSP register.\\
\hline
r8 & uint64 & The R8 register.\\
\hline
r9 & uint64 & The R9 register.\\
\hline
r10 & uint64 & The R10 register.\\
\hline
r11 & uint64 & The R11 register.\\
\hline
r12 & uint64 & The R12 register.\\
\hline
r13 & uint64 & The R13 register.\\
\hline
r14 & uint64 & The R14 register.\\
\hline
r15 & uint64 & The R15 register.\\
\hline
cs & uint64 & The CS register.\\
\hline
ds & uint64 & The DS register.\\
\hline
ss & uint64 & The SS register.\\
\hline
es & uint64 & The ES register.\\
\hline
fs & uint64 & The FS register.\\
\hline
gs & uint64 & The GS register.\\
\hline
rflags & uint64 & The RFLAGS register.\\
\hline
eip & uint64 & The EIP register.\\
\hline
cr0 & uint64 & The CR0 register.\\
\hline
cr1 & uint64 & The CR1 register.\\
\hline
cr2 & uint64 & The CR2 register.\\
\hline
cr3 & uint64 & The CR3 register.\\
\hline
cr4 & uint64 & The CR4 register.\\
\hline
cr8 & uint64 & The CR8 register.\\
\hline
gdtr\_0 & uint64 & The first \texttt{UINT64} of the GDTR register.\\
\hline
gdtr\_1 & uint64 & The second \texttt{UINT64} of the GDTR register.\\
\hline
idtr\_0 & uint64 & The first \texttt{UINT64} of the IDTR register.\\
\hline
idtr\_1 & uint64 & The second \texttt{UINT64} of the IDTR register.\\
\hline
ldtr & uint64 & The LDTR register.\\
\hline
tr & uint64 & The TR register.\\
\jsontableend{IA32/x64 x64 Register State structure field table.}

% IA32/x64 IA32 Register State structure
\subsection{IA32/x64 Unformatted Register State Structure}
\label{subsection:ia32x64unformattedregisterstatestructure}
This structure describes a single IA32/x64 unformatted register state, which is contained in IA32/x64 Processor Context Info structures (\ref{subsection:ia32x64processorcontextinfostructure}) when\\\texttt{registerContextType.value} has a value other than "2" or "3".
\jsontable{table:ia32x64unformattedregisterstatestructure}
data & string & A base64-formatted binary representation of the register array.\\
\jsontableend{IA32/x64 Unformatted Register State structure field table.}

% ARM processor error section.
\section{ARM Processor Error Section}
\label{section:armprocessorerrorsection}
This section describes the JSON format for a single ARM Processor Error Section from a CPER record. The GUID used for ARM Processor Error Sections is \texttt{\{ 0xe19e3d16, 0xbc11, 0x11e4, \{ 0x9c, 0xaa, 0xc2, 0x05, 0x1d, 0x5d, 0x46, 0xb0 \}\}}.
\jsontable{table:armprocessorerrorsection}
validationBits & object & An ARM Processor Error Validation structure, as defined in Subsection .\\
\hline
errorInfoNum & int & The number of error info structures attached to this error.\\
\hline
contextInfoNum & int & The number of context info structures attached to this error.\\
\hline
sectionLength & uint64 & The total size (in bytes) of this error section.\\
\hline
errorAffinity.value & int & The raw value of the error affinity for this error.\\
errorAffinity.type & string & The human readable type of the error affinity for this error. All values are vendor defined, so specific names cannot be provided.\\
\hline
mpidrEl1 & uint64 & The processor ID (\texttt{MPIDR\_EL1}) for this error.\\
\hline
midrEl1 & uint64 & The chip ID (\texttt{MIDR\_EL1}) for this error.\\
\hline
running & boolean & Whether the processor is running or not. If true, the \texttt{psciState} field is not included.\\
\hline
psciState & uint64 (\textbf{optional}) & The PSCI state of the processor. Only \textbf{optionally} included when the "running" field is false. Cannot be made human readable, as this could either be in the pre-PSCI 1.0 format, or the newer "Extended StateID" format. For more information, see the ARM PSCI specification.\\
\hline
errorInfo & array & Array of ARM Processor Error Info structures, as defined in Subsection \ref{subsection:armprocessorerrorinfostructure}.\\
\hline
contextInfo & array & Array of ARM Processor Context Info structures, as defined in Subsection \ref{subsection:armprocessorcontextinfostructure}.\\
\hline
vendorSpecificInfo.data & string & A base64-encoded binary representation of any attached vendor specific information.\\
\jsontableend{ARM Processor Error structure field table.}

% ARM Processor Error Validation structure
\subsection{ARM Processor Error Validation Structure}
\label{subsection:armprocessorerrorvalidationstructure}
This structure describes which fields are valid in a single ARM Processor Error structure (\ref{section:armprocessorerrorsection}) with boolean fields.
\jsontable{table:armprocessorerrorvalidationstructure}
mpidrValid & boolean & Whether the "mpidrEl1" field in the ARM Processor Error structure (\ref{section:armprocessorerrorsection}) is valid.\\
\hline
errorAffinityLevelValid & boolean & Whether the "errorAffinity" field in the ARM Processor Error structure (\ref{section:armprocessorerrorsection}) is valid.\\
\hline
runningStateValid & boolean & Whether the "running" field in the ARM Processor Error structure (\ref{section:armprocessorerrorsection}) is valid.\\
\hline
vendorSpecificInfoValid & boolean & Whether the trailing vendor specific info (if present) in the ARM Processor Error Structure (\ref{section:armprocessorerrorsection}) is valid.\\
\jsontableend{ARM Processor Error validation structure field table.}

% ARM Processor Error Info structure
\subsection{ARM Processor Error Info Structure}
\label{subsection:armprocessorerrorinfostructure}
This structure describes a single ARM Processor Error Info structure, as part of a whole ARM Processor Error structure (\ref{section:armprocessorerrorsection}).
\jsontable{table:armprocessorerrorinfostructure}
version & int & The version of the structure that is implemented.\\
\hline
length & int & The length of the structure, in bytes. For version 0, this is 32.\\
\hline
validationBits & object & An ARM Processor Error Info Validation structure as defined in Subsection \ref{subsection:armprocessorerrorinfovalidationstructure}.\\
\hline
errorType.value & uint64 & The raw value of the error type this error info describes.\\
errorType.name & string & The human readable name, if available, of the error type this error info describes.\\
\hline
multipleError.value & int & If the value of this field is 2 or greater, the raw value of the number of errors that occurred. Otherwise, the raw value of the multiple error status.\\
multipleError.type & string & The human readable value, if available, of what type of multiple error this is (single error, multiple error).\\
\hline
flags & object & An ARM Processor Error Info Flags structure as defined in Subsection \ref{subsection:armprocessorerrorinfoflagsstructure}.\\ 
\hline
errorInformation & object & An error information structure, as defined in one of Subsections \ref{subsection:armprocessorerrorinfoerrorinformationcachetlbstructure} or \ref{subsection:armprocessorerrorinfoerrorinformationbusstructure}. Which structure this is depends on the \texttt{errorType.value} field.\\
\hline
virtualFaultAddress & uint64 & Indicates a virtual fault address associated with the error, such as when an error occurs in virtually indexed cache.\\
\hline
physicalFaultAddress & uint64 & Indicates a physical fault address associated with the error.\\
\jsontableend{ARM Processor Error Info structure field table.}

% ARM Processor Error Info Validation structure
\subsection{ARM Processor Error Info Validation Structure}
\label{subsection:armprocessorerrorinfovalidationstructure}
This structure describes the valid fields in a single ARM Processor Error Info structure (\ref{subsection:armprocessorerrorinfostructure}), using boolean fields.
\jsontable{table:armprocessorerrorinfovalidationstructure}
multipleErrorValid & boolean & Whether the "multipleError" field in the ARM Processor Error Info structure (\ref{subsection:armprocessorerrorinfostructure}) is valid.\\
\hline
flagsValid & boolean & Whether the "flags" field in the ARM Processor Error Info structure (\ref{subsection:armprocessorerrorinfostructure}) is valid.\\
\hline
errorInformationValid & boolean & Whether the "errorInformation" field in the ARM Processor Error Info structure (\ref{subsection:armprocessorerrorinfostructure}) is valid.\\
\hline
virtualFaultAddressValid & boolean & Whether the "virtualFaultAddress" field in the ARM Processor Error Info structure (\ref{subsection:armprocessorerrorinfostructure}) is valid.\\
\hline
physicalFaultAddressValid & boolean & Whether the "physicalFaultAddress" field in the ARM Processor Error Info structure (\ref{subsection:armprocessorerrorinfostructure}) is valid.\\
\jsontableend{ARM Processor Error Info validation structure field table.}

% ARM Processor Error Info Validation structure
\subsection{ARM Processor Error Info Flags Structure}
\label{subsection:armprocessorerrorinfoflagsstructure}
This structure describes the flags in a single ARM Processor Error Info structure (\ref{subsection:armprocessorerrorinfostructure}), using boolean fields.
\jsontable{table:armprocessorerrorinfoflagsstructure}
firstErrorCaptured & boolean & Whether this is the first error captured.\\
\hline
lastErrorCaptured & boolean & Whether this is the last error captured.\\
\hline
propagated & boolean & Whether the error has propagated.\\
\hline
overflow & boolean & Whether error buffer overflow was detected. This is usually from multiple errors occurring in a short timespan, and indicates loss of error data.\\
\jsontableend{ARM Processor Error Info Flags structure field table.}

% ARM Processor Error Info Error Information (Cache/TLB) structure
\subsection{ARM Processor Error Info Cache/TLB Information Structure}
\label{subsection:armprocessorerrorinfoerrorinformationcachetlbstructure}
This structure describes cache/TLB error information for a single ARM Processor Error Info structure (\ref{subsection:armprocessorerrorinfostructure}).
\jsontable{table:armprocessorerrorinfoerrorinformationcachetlbstructure}
validationBits & object & An ARM Processor Info Cache/TLB Validation structure as defined in Subsection \ref{subsection:armprocessorerrorinfocachetlbvalidationstructure}.\\
\hline
transactionType.value & uint64 & The raw value of the type of cache/TLB error.\\
transactionType.name & string & The human readable name, if available, of the type of cache/TLB error.\\
\hline
operation.value & uint64 & The raw value of the cache/TLB operation that caused the error.\\
operation.name & string & The human readable name, if available, of the cache/TLB operation that caused the error.\\
\hline
level & int & The cache/TLB level that the error occurred at.\\
\hline
processorContextCorrupt & boolean & Whether the processor context may have been corrupted.\\
\hline
corrected & boolean & Whether the error was corrected.\\
\hline
precisePC & boolean & Whether the program counter is directly associated with the error.\\
\hline
restartablePC & boolean & Whether program execution can be restarted reliably at the program counter associated with the error.\\
\jsontableend{ARM Processor Error Info Cache/TLB Information structure field table.}

% ARM Processor Error Info Error Information (Cache/TLB) validation structure
\subsection{ARM Processor Error Info Cache/TLB Validation Structure}
\label{subsection:armprocessorerrorinfocachetlbvalidationstructure}
This structure describes valid fields in a single ARM Processor Error Info Cache/TLB Information structure (\ref{subsection:armprocessorerrorinfoerrorinformationcachetlbstructure}), as a set of boolean fields.
\jsontable{table:armprocessorerrorinfocachetlbvalidationstructure}
transactionTypeValid & boolean & Whether the "transactionType" field in the ARM Processor Info Cache/TLB Information structure (\ref{subsection:armprocessorerrorinfoerrorinformationcachetlbstructure}) is valid.\\
\hline
operationValid & boolean & Whether the "operation" field in the ARM Processor Info Cache/TLB Information structure (\ref{subsection:armprocessorerrorinfoerrorinformationcachetlbstructure}) is valid.\\
\hline
levelValid & boolean & Whether the "level" field in the ARM Processor Info Cache/TLB Information structure (\ref{subsection:armprocessorerrorinfoerrorinformationcachetlbstructure}) is valid.\\
\hline
processorContextCorruptValid & boolean & Whether the "processorContextCorrupt" field in the ARM Processor Info Cache/TLB Information structure (\ref{subsection:armprocessorerrorinfoerrorinformationcachetlbstructure}) is valid.\\
\hline
correctedValid & boolean & Whether the "corrected" field in the ARM Processor Info Cache/TLB Information structure (\ref{subsection:armprocessorerrorinfoerrorinformationcachetlbstructure}) is valid.\\
\hline
precisePCValid & boolean & Whether the "precisePC" field in the ARM Processor Info Cache/TLB Information structure (\ref{subsection:armprocessorerrorinfoerrorinformationcachetlbstructure}) is valid.\\
\hline
restartablePCValid & boolean & Whether the "restartablePC" field in the ARM Processor Info Cache/TLB Information structure (\ref{subsection:armprocessorerrorinfoerrorinformationcachetlbstructure}) is valid.\\
\jsontableend{ARM Processor Error Info Cache/TLB validation structure field table.}

% ARM Processor Error Info Error Information (Bus) structure
\subsection{ARM Processor Error Info Bus Information Structure}
\label{subsection:armprocessorerrorinfoerrorinformationbusstructure}
This structure describes bus error information for a single ARM Processor Error Info structure (\ref{subsection:armprocessorerrorinfostructure}).
\jsontable{table:armprocessorerrorinfoerrorinformationbusstructure}
validationBits & object & An ARM Processor Info Bus Validation structure as defined in Subsection \ref{subsection:armprocessorerrorinfobusvalidationstructure}.\\
\hline
transactionType.value & uint64 & The raw value of the type of bus error.\\
transactionType.name & string & The human readable name, if available, of the type of bus error.\\
\hline
operation.value & uint64 & The raw value of the bus operation that caused the error.\\
operation.name & string & The human readable name, if available, of the bus operation that caused the error.\\
\hline
level & int & The affinity level that the bus error occurred at.\\
\hline
processorContextCorrupt & boolean & Whether the processor context may have been corrupted.\\
\hline
corrected & boolean & Whether the error was corrected.\\
\hline
precisePC & boolean & Whether the program counter is directly associated with the error.\\
\hline
restartablePC & boolean & Whether program execution can be restarted reliably at the program counter associated with the error.\\
\hline
timedOut & boolean & Whether the request timed out.\\
\hline
participationType.value & uint64 & The raw value of the type of participation that occurred in the bus error.\\
participationType.name & string & The human readable name, if available, of the type of participation that occurred in the bus error.\\
\hline
addressSpace.value & uint64 & The raw value of the address space in which the bus error occurred.\\
addressSpace.name & string & The human readable name, if available, of the address space in which the bus error occurred.\\
\hline
memoryAttributes & int & Memory access attributes for this bus error as described in the ARM ARM.\\
\hline
accessMode.value & int & The raw value of the access mode of the bus request (secure/normal).\\
accessMode.name & string & The human readable name, if available, of the access mode of the bus request (secure/normal).\\
\jsontableend{ARM Processor Error Info Bus Information structure field table.}

% ARM Processor Error Info Error Information (Bus) validation structure
\subsection{ARM Processor Error Info Bus Validation Structure}
\label{subsection:armprocessorerrorinfobusvalidationstructure}
This structure describes valid fields in a single ARM Processor Error Info Cache/TLB Information structure (\ref{subsection:armprocessorerrorinfoerrorinformationbusstructure}), as a set of boolean fields.
\jsontable{table:armprocessorerrorinfobusvalidationstructure}
transactionTypeValid & boolean & Whether the "transactionType" field in the ARM Processor Info Bus Information structure (\ref{subsection:armprocessorerrorinfoerrorinformationbusstructure}) is valid.\\
\hline
operationValid & boolean & Whether the "operation" field in the ARM Processor Info Bus Information structure (\ref{subsection:armprocessorerrorinfoerrorinformationbusstructure}) is valid.\\
\hline
levelValid & boolean & Whether the "level" field in the ARM Processor Info Bus Information structure (\ref{subsection:armprocessorerrorinfoerrorinformationbusstructure}) is valid.\\
\hline
processorContextCorruptValid & boolean & Whether the "processorContextCorrupt" field in the ARM Processor Info Bus Information structure (\ref{subsection:armprocessorerrorinfoerrorinformationbusstructure}) is valid.\\
\hline
correctedValid & boolean & Whether the "corrected" field in the ARM Processor Info Bus Information structure (\ref{subsection:armprocessorerrorinfoerrorinformationbusstructure}) is valid.\\
\hline
precisePCValid & boolean & Whether the "precisePC" field in the ARM Processor Info Bus Information structure (\ref{subsection:armprocessorerrorinfoerrorinformationbusstructure}) is valid.\\
\hline
restartablePCValid & boolean & Whether the "restartablePC" field in the ARM Processor Info Bus Information structure (\ref{subsection:armprocessorerrorinfoerrorinformationbusstructure}) is valid.\\
\hline
participationTypeValid & boolean & Whether the "participationType" field in the ARM Processor Info Bus Information structure (\ref{subsection:armprocessorerrorinfoerrorinformationbusstructure}) is valid.\\
\hline
timedOutValid & boolean & Whether the "timedOut" field in the ARM Processor Info Bus Information structure (\ref{subsection:armprocessorerrorinfoerrorinformationbusstructure}) is valid.\\
\hline
addressSpaceValid & boolean & Whether the "addressSpace" field in the ARM Processor Info Bus Information structure (\ref{subsection:armprocessorerrorinfoerrorinformationbusstructure}) is valid.\\
\hline
memoryAttributesValid & boolean & Whether the "memoryAttributes" field in the ARM Processor Info Bus Information structure (\ref{subsection:armprocessorerrorinfoerrorinformationbusstructure}) is valid.\\
\hline
accessModeValid & boolean & Whether the "accessMode" field in the ARM Processor Info Bus Information structure (\ref{subsection:armprocessorerrorinfoerrorinformationbusstructure}) is valid.\\
\jsontableend{ARM Processor Error Info Bus validation structure field table.}

% ARM Processor Context Info structure
\subsection{ARM Processor Context Info Structure}
\label{subsection:armprocessorcontextinfostructure}
This structure describes a single ARM Processor Context Info structure, as part of a whole ARM Processor Error structure (\ref{section:armprocessorerrorsection}).
\jsontable{table:armprocessorcontextinfostructure}
registerContextType.value & uint64 & The raw value of the type of processor context state being reported.\\
registerContextType.name & string & The human readable name, if available, of the type of processor context state being reported.\\
\hline
registerArraySize & uint64 & The size of the attached register array, in bytes.\\
\hline
registerArray & object & The attached register array, with registers encoded as object fields. Structured as shown in one of subsections \ref{subsection:armaarch32gprstructure}, \ref{subsection:armaarch32el1contextregistersstructure}, \ref{subsection:armaarch32el2contextregistersstructure}, \ref{subsection:armaarch32secureregistersstructure}, \ref{subsection:armaarch64gprstructure}, \ref{subsection:armaarch64el1contextregistersstructure}, \ref{subsection:armaarch64el2contextregistersstructure}, \ref{subsection:armaarch64el3contextregistersstructure}, \ref{subsection:armmiscregistersstructure} or \ref{subsection:armunknownregistersstructure}. Type of structure depends on the \texttt{registerContextType.value} field.\\
\jsontableend{ARM Processor Context Info structure field table.}

% ARM AARCH32 General Purpose Registers structure
\subsection{ARM AARCH32 General Purpose Registers Structure}
\label{subsection:armaarch32gprstructure}
This structure describes the register array for AARCH32 GPRs as part of an ARM Processor Context Info Structure (\ref{subsection:armprocessorcontextinfostructure}). This structure is included when the field \texttt{registerContextType.value} has the value 0.
\jsontable{table:armaarch32gprstructure}
r0 & uint64 & Register R0. \texttt{UINT32} value null extended to \texttt{UINT64}.\\
\hline
r1 & uint64 & Register R1. \texttt{UINT32} value null extended to \texttt{UINT64}.\\
\hline
r2 & uint64 & Register R2. \texttt{UINT32} value null extended to \texttt{UINT64}.\\
\hline
r3 & uint64 & Register R3. \texttt{UINT32} value null extended to \texttt{UINT64}.\\
\hline
r4 & uint64 & Register R4. \texttt{UINT32} value null extended to \texttt{UINT64}.\\
\hline
r5 & uint64 & Register R5. \texttt{UINT32} value null extended to \texttt{UINT64}.\\
\hline
r6 & uint64 & Register R6. \texttt{UINT32} value null extended to \texttt{UINT64}.\\
\hline
r7 & uint64 & Register R7. \texttt{UINT32} value null extended to \texttt{UINT64}.\\
\hline
r8 & uint64 & Register R8. \texttt{UINT32} value null extended to \texttt{UINT64}.\\
\hline
r9 & uint64 & Register R9. \texttt{UINT32} value null extended to \texttt{UINT64}.\\
\hline
r10 & uint64 & Register R10. \texttt{UINT32} value null extended to \texttt{UINT64}.\\
\hline
r11 & uint64 & Register R11. \texttt{UINT32} value null extended to \texttt{UINT64}.\\
\hline
r12 & uint64 & Register R12. \texttt{UINT32} value null extended to \texttt{UINT64}.\\
\hline
r13\_sp & uint64 & Register R13 (SP). \texttt{UINT32} value null extended to \texttt{UINT64}.\\
\hline
r14\_lr & uint64 & Register R14 (LR). \texttt{UINT32} value null extended to \texttt{UINT64}.\\
\hline
r15\_pc & uint64 & Register R15 (PC). \texttt{UINT32} value null extended to \texttt{UINT64}.\\
\jsontableend{ARM AARCH32 General Purpose Registers structure field table.}

% ARM AARCH32 EL1 Context Registers structure
\subsection{ARM AARCH32 EL1 Context Registers Structure}
\label{subsection:armaarch32el1contextregistersstructure}
This structure describes the register array for AARCH32 EL1 context registers as part of an ARM Processor Context Info Structure (\ref{subsection:armprocessorcontextinfostructure}). This structure is included when the field \texttt{registerContextType.value} has the value 1.
\jsontable{table:armaarch32el1contextregistersstructure}
dfar & uint64 & Register DFAR. \texttt{UINT32} value null extended to \texttt{UINT64}.\\
\hline
dfsr & uint64 & Register DFSR. \texttt{UINT32} value null extended to \texttt{UINT64}.\\
\hline
ifar & uint64 & Register IFAR. \texttt{UINT32} value null extended to \texttt{UINT64}.\\
\hline
isr & uint64 & Register ISR. \texttt{UINT32} value null extended to \texttt{UINT64}.\\
\hline
mair0 & uint64 & Register MAIR0. \texttt{UINT32} value null extended to \texttt{UINT64}.\\
\hline
mair1 & uint64 & Register MAIR1. \texttt{UINT32} value null extended to \texttt{UINT64}.\\
\hline
midr & uint64 & Register MIDR. \texttt{UINT32} value null extended to \texttt{UINT64}.\\
\hline
mpidr & uint64 & Register MPIDR. \texttt{UINT32} value null extended to \texttt{UINT64}.\\
\hline
nmrr & uint64 & Register NMRR. \texttt{UINT32} value null extended to \texttt{UINT64}.\\
\hline
prrr & uint64 & Register PRRR. \texttt{UINT32} value null extended to \texttt{UINT64}.\\
\hline
sctlr\_ns & uint64 & Register SCTLR (NS). \texttt{UINT32} value null extended to \texttt{UINT64}.\\
\hline
spsr & uint64 & Register SPSR. \texttt{UINT32} value null extended to \texttt{UINT64}.\\
\hline
spsr\_abt & uint64 & Register SPSR (ABT). \texttt{UINT32} value null extended to \texttt{UINT64}.\\
\hline
spsr\_fiq & uint64 & Register SPSR (FIQ). \texttt{UINT32} value null extended to \texttt{UINT64}.\\
\hline
spsr\_irq & uint64 & Register SPSR (IRQ). \texttt{UINT32} value null extended to \texttt{UINT64}.\\
\hline
spsr\_svc & uint64 & Register SPSR (SVC). \texttt{UINT32} value null extended to \texttt{UINT64}.\\
\hline
spsr\_und & uint64 & Register SPSR (UND). \texttt{UINT32} value null extended to \texttt{UINT64}.\\
\hline
tpidrprw & uint64 & Register TPIDR (PRW). \texttt{UINT32} value null extended to \texttt{UINT64}.\\
\hline
tpidruro & uint64 & Register TPIDR (URO). \texttt{UINT32} value null extended to \texttt{UINT64}.\\
\hline
tpidrurw & uint64 & Register TPIDR (URW). \texttt{UINT32} value null extended to \texttt{UINT64}.\\
\hline
ttbcr & uint64 & Register TTBCR. \texttt{UINT32} value null extended to \texttt{UINT64}.\\
\hline
ttbr0 & uint64 & Register TTBR0. \texttt{UINT32} value null extended to \texttt{UINT64}.\\
\hline
ttbr1 & uint64 & Register TTBR1. \texttt{UINT32} value null extended to \texttt{UINT64}.\\
\hline
dacr & uint64 & Register DACR. \texttt{UINT32} value null extended to \texttt{UINT64}.\\
\jsontableend{ARM AARCH32 EL1 Context Registers structure field table.}

% ARM AARCH32 EL2 Context Registers structure
\subsection{ARM AARCH32 EL2 Context Registers Structure}
\label{subsection:armaarch32el2contextregistersstructure}
This structure describes the register array for AARCH32 EL2 context registers as part of an ARM Processor Context Info Structure (\ref{subsection:armprocessorcontextinfostructure}). This structure is included when the field \texttt{registerContextType.value} has the value 2.
\jsontable{table:armaarch32el2contextregistersstructure}
elr\_hyp & uint64 & Register ELR\_HYP. \texttt{UINT32} value null extended to \texttt{UINT64}.\\
\hline
hamair0 & uint64 & Register HAMAIR0. \texttt{UINT32} value null extended to \texttt{UINT64}.\\
\hline
hamair1 & uint64 & Register HAMAIR1. \texttt{UINT32} value null extended to \texttt{UINT64}.\\
\hline
hcr & uint64 & Register HCR. \texttt{UINT32} value null extended to \texttt{UINT64}.\\
\hline
hcr2 & uint64 & Register HCR2. \texttt{UINT32} value null extended to \texttt{UINT64}.\\
\hline
hdfar & uint64 & Register HDFAR. \texttt{UINT32} value null extended to \texttt{UINT64}.\\
\hline
hifar & uint64 & Register HIFAR. \texttt{UINT32} value null extended to \texttt{UINT64}.\\
\hline
hpfar & uint64 & Register HPFAR. \texttt{UINT32} value null extended to \texttt{UINT64}.\\
\hline
hsr & uint64 & Register HSR. \texttt{UINT32} value null extended to \texttt{UINT64}.\\
\hline
htcr & uint64 & Register HTCR. \texttt{UINT32} value null extended to \texttt{UINT64}.\\
\hline
htpidr & uint64 & Register HTPIDR. \texttt{UINT32} value null extended to \texttt{UINT64}.\\
\hline
httbr & uint64 & Register HTTBR. \texttt{UINT32} value null extended to \texttt{UINT64}.\\
\hline
spsr\_hyp & uint64 & Register SPSR (HYP). \texttt{UINT32} value null extended to \texttt{UINT64}.\\
\hline
vtcr & uint64 & Register VTCR. \texttt{UINT32} value null extended to \texttt{UINT64}.\\
\hline
vttbr & uint64 & Register VTTBR. \texttt{UINT32} value null extended to \texttt{UINT64}.\\
\hline
dacr32\_el2 & uint64 & Register DACR32 (EL2). \texttt{UINT32} value null extended to \texttt{UINT64}.\\
\hline
\jsontableend{ARM AARCH32 EL2 Context Registers structure field table.}

% ARM AARCH32 Secure Registers structure
\subsection{ARM AARCH32 Secure Registers Structure}
\label{subsection:armaarch32secureregistersstructure}
This structure describes the register array for AARCH32 secure registers as part of an ARM Processor Context Info Structure (\ref{subsection:armprocessorcontextinfostructure}). This structure is included when the field \texttt{registerContextType.value} has the value 3.
\jsontable{table:armaarch32secureregistersstructure}
sctlr\_s & uint64 & Register SCTLR\_S. \texttt{UINT32} value null extended to \texttt{UINT64}.\\
\hline
spsr\_mon & uint64 & Register SPSR (MON). \texttt{UINT32} value null extended to \texttt{UINT64}.\\
\jsontableend{ARM AARCH32 Secure Registers structure field table.}

% ARM AARCH64 General Purpose Registers structure
\subsection{ARM AARCH64 General Purpose Registers Structure}
\label{subsection:armaarch64gprstructure}
This structure describes the register array for AARCH64 GPRs as part of an ARM Processor Context Info Structure (\ref{subsection:armprocessorcontextinfostructure}). This structure is included when the field \texttt{registerContextType.value} has the value 4.
\jsontable{table:armaarch64gprstructure}
x0 & uint64 & Register X0.\\
\hline
x1 & uint64 & Register X1.\\
\hline
x2 & uint64 & Register X2.\\
\hline
x3 & uint64 & Register X3.\\
\hline
x4 & uint64 & Register X4.\\
\hline
x5 & uint64 & Register X5.\\
\hline
x6 & uint64 & Register X6.\\
\hline
x7 & uint64 & Register X7.\\
\hline
x8 & uint64 & Register X8.\\
\hline
x9 & uint64 & Register X9.\\
\hline
x10 & uint64 & Register X10.\\
\hline
x11 & uint64 & Register X11.\\
\hline
x12 & uint64 & Register X12.\\
\hline
x13 & uint64 & Register X13.\\
\hline
x14 & uint64 & Register X14.\\
\hline
x15 & uint64 & Register X15.\\
\hline
x16 & uint64 & Register X16.\\
\hline
x17 & uint64 & Register X17.\\
\hline
x18 & uint64 & Register X18.\\
\hline
x19 & uint64 & Register X19.\\
\hline
x20 & uint64 & Register X20.\\
\hline
x21 & uint64 & Register X21.\\
\hline
x22 & uint64 & Register X22.\\
\hline
x23 & uint64 & Register X23.\\
\hline
x24 & uint64 & Register X24.\\
\hline
x25 & uint64 & Register X25.\\
\hline
x26 & uint64 & Register X26.\\
\hline
x27 & uint64 & Register X27.\\
\hline
x28 & uint64 & Register X28.\\
\hline
x29 & uint64 & Register X29.\\
\hline
x30 & uint64 & Register X30.\\
\hline
sp & uint64 & Register SP.\\
\jsontableend{ARM AARCH64 General Purpose Registers structure field table.}

% ARM AARCH64 EL1 Context Registers structure
\subsection{ARM AARCH64 EL1 Context Registers Structure}
\label{subsection:armaarch64el1contextregistersstructure}
This structure describes the register array for AARCH64 EL1 context registers as part of an ARM Processor Context Info Structure (\ref{subsection:armprocessorcontextinfostructure}). This structure is included when the field \texttt{registerContextType.value} has the value 5.
\jsontable{table:armaarch64el1contextregistersstructure}
elr\_el1 & uint64 & Register ELR (EL1).\\
\hline
esr\_el1 & uint64 & Register ESR (EL1).\\
\hline
far\_el1 & uint64 & Register FAR (EL1).\\
\hline
isr\_el1 & uint64 & Register ISR (EL1).\\
\hline
mair\_el1 & uint64 & Register MAIR (EL1).\\
\hline
midr\_el1 & uint64 & Register MIDR (EL1).\\
\hline
mpidr\_el1 & uint64 & Register MPIDR (EL1).\\
\hline
sctlr\_el1 & uint64 & Register SCTLR (EL1).\\
\hline
sp\_el0 & uint64 & Register SP (EL0).\\
\hline
sp\_el1 & uint64 & Register SP (EL1).\\
\hline
spsr\_el1 & uint64 & Register SPSR (EL1).\\
\hline
tcr\_el1 & uint64 & Register TCR (EL1).\\
\hline
tpidr\_el0 & uint64 & Register TPIDR (EL0).\\
\hline
tpidr\_el1 & uint64 & Register TPIDR (EL1).\\
\hline
tpidrro\_el0 & uint64 & Register TPIDRRO (EL0).\\
\hline
ttbr0\_el1 & uint64 & Register TTBR0 (EL1).\\
\hline
ttbr1\_el1 & uint64 & Register TTBR1 (EL1).\\
\jsontableend{ARM AARCH64 EL1 Context Registers structure field table.}

% ARM AARCH64 EL2 Context Registers structure
\subsection{ARM AARCH64 EL2 Context Registers Structure}
\label{subsection:armaarch64el2contextregistersstructure}
This structure describes the register array for AARCH64 EL2 context registers as part of an ARM Processor Context Info Structure (\ref{subsection:armprocessorcontextinfostructure}). This structure is included when the field \texttt{registerContextType.value} has the value 6.
\jsontable{table:armaarch64el2contextregistersstructure}
elr\_el2 & uint64 & Register ELR (EL2).\\
\hline
esr\_el2 & uint64 & Register ESR (EL2).\\
\hline
far\_el2 & uint64 & Register FAR (EL2).\\
\hline
hacr\_el2 & uint64 & Register HACR (EL2).\\
\hline
hcr\_el2 & uint64 & Register HCR (EL2).\\
\hline
hpfar\_el2 & uint64 & Register HPFAR (EL2).\\
\hline
mair\_el2 & uint64 & Register MAIR (EL2).\\
\hline
sctlr\_el2 & uint64 & Register SCTLR (EL2).\\
\hline
sp\_el2 & uint64 & Register SP (EL2).\\
\hline
spsr\_el2 & uint64 & Register SPSR (EL2).\\
\hline
tcr\_el2 & uint64 & Register TCR (EL2).\\
\hline
tpidr\_el2 & uint64 & Register TPIDR (EL2).\\
\hline
ttbr0\_el2 & uint64 & Register TTBR0 (EL2).\\
\hline
vtcr\_el2 & uint64 & Register VTCR (EL2).\\
\hline
vttbr\_el2 & uint64 & Register VTTBR (EL2).\\
\jsontableend{ARM AARCH64 EL2 Context Registers structure field table.}

% ARM AARCH64 EL3 Context Registers structure
\subsection{ARM AARCH64 EL3 Context Registers Structure}
\label{subsection:armaarch64el3contextregistersstructure}
This structure describes the register array for AARCH64 EL3 context registers as part of an ARM Processor Context Info Structure (\ref{subsection:armprocessorcontextinfostructure}). This structure is included when the field \texttt{registerContextType.value} has the value 7.
\jsontable{table:armaarch64el3contextregistersstructure}
elr\_el3 & uint64 & Register ELR (EL3).\\
\hline
esr\_el3 & uint64 & Register ESR (EL3).\\
\hline
far\_el3 & uint64 & Register FAR (EL3).\\
\hline
mair\_el3 & uint64 & Register MAIR (EL3).\\
\hline
sctlr\_el3 & uint64 & Register SCTLR (EL3).\\
\hline
sp\_el3 & uint64 & Register SP (EL3).\\
\hline
spsr\_el3 & uint64 & Register SPSR (EL3).\\
\hline
tcr\_el3 & uint64 & Register TCR (EL3).\\
\hline
tpidr\_el3 & uint64 & Register TPIDR (EL3).\\
\hline
ttbr0\_el3 & uint64 & Register TTBR0 (EL3).\\
\jsontableend{ARM AARCH64 EL3 Context Registers structure field table.}

% ARM AARCH64 Miscellaneous Registers structure
\subsection{ARM AARCH64 Miscellaneous Registers Structure}
\label{subsection:armmiscregistersstructure}
This structure describes the register array for miscellaneous ARM registers as part of an ARM Processor Context Info Structure (\ref{subsection:armprocessorcontextinfostructure}). This structure is included when the field \texttt{registerContextType.value} has the value 8.
\jsontable{table:armmiscregistersstructure}
mrsEncoding.op2 & uint64 & MRS Encoding OP2.\\
\hline
mrsEncoding.crm & uint64 & MRS Encoding CRm.\\
\hline
mrsEncoding.crn & uint64 & MRS Encoding CRn.\\
\hline
mrsEncoding.op1 & uint64 & MRS Encoding Op1.\\
\hline
mrsEncoding.o0 & uint64 & MRS Encoding O0.\\
\hline
value & uint64 & Value of the single register.\\
\jsontableend{ARM AARCH64 Miscellaneous Registers structure field table.}

% ARM AARCH64 Unknown Registers structure
\subsection{ARM AARCH64 Unknown Registers Structure}
\label{subsection:armunknownregistersstructure}
This structure describes the register array for unknown ARM registers as part of an ARM Processor Context Info Structure (\ref{subsection:armprocessorcontextinfostructure}). This structure is included when the field \texttt{registerContextType.value} has any value other than 0-8 (inclusive).
\jsontable{table:armunknownregistersstructure}
data & string & A base64 representation of the unknown binary register array data.\\
\jsontableend{ARM AARCH64 Unknown Registers structure field table.}

% Memory error section.
\section{Memory Error Section}
\label{section:memoryerrorsection}
This section describes the JSON format for a single Memory Error Section from a CPER record. The GUID used for Memory Error Sections is \texttt{\{ 0xa5bc1114, 0x6f64, 0x4ede, \{ 0xb8, 0x63, 0x3e, 0x83, 0xed, 0x7c, 0x83, 0xb1 \}\}}.
\jsontable{table:memoryerrorsection}
validationBits & object & A Memory Error Validation structure, as described in Subsection \ref{subsection:memoryerrorvalidationstructure}.\\
\hline
errorStatus & object & A CPER Generic Error Status structure, as described in Subsection \ref{subsection:genericerrorstatusstructure}.\\
\hline
bank & object & Structure as described in one of Subsection \ref{subsection:memoryerrorstandardbankaddressstructure} or Subsection \ref{subsection:memoryerroraddressgroupbankaddressstructure}. Selected structure depends on the \texttt{validationBits.bankValid} field.\\
\hline
memoryErrorType.value & uint64 & The raw value of the memory error type.\\
memoryErrorType.name & string & The human readable name, if available, of the memory error type.\\
\hline
extended.rowBit16 & boolean & Bit 16 of the row number of the memory error location.\\
extended.rowBit17 & boolean & Bit 17 of the row number of the memory error location.\\
extended.chipIdentification & int & The ID of the related chip.\\
\hline
physicalAddress & uint64 & The physical address at which the error occurred.\\
\hline
physicalAddressMask & uint64 & Defines the valid address bits in the \texttt{physicalAddress} field.\\
\hline
node & uint64 & Identifies the node containing the memory error, if in a multi-node system.\\
\hline
card & uint64 & The card number of the memory error location.\\
\hline
moduleRank & uint64 & The module or rank number of the offending memory error location.\\
\hline
device & uint64 & The device number of the memory associated with the error.\\
\hline
row & uint64 & The first 16 bits of the row number of the memory location.\\
\hline
column & uint64 & The column number of the memory error location.\\
\hline
bitPosition & uint64 & The bit position at which the error occurred.\\
\hline
requestorID & uint64 & Hardware address of the device that initiated the errored transaction.\\
\hline
responderID & uint64 & Hardware address of the device that responded to the transaction.\\
\hline
targetID & uint64 & Hardware address of the intended target of the transaction.\\
\hline
rankNumber & uint64 & The rank number of the memory error location.\\
\hline
cardSmbiosHandle & uint64 & The SMBIOS handle for the memory card's Type 16 Memory Array Structure.\\
\hline
moduleSmbiosHandle & uint64 & The SMBIOS handle for the memory module's Type 17 Memory Device Structure.\\
\jsontableend{Memory Error structure field table.}

% Memory error validation structure.
\subsection{Memory Error Validation Structure}
\label{subsection:memoryerrorvalidationstructure}
This structure describes whether fields in a single Memory Error (\ref{section:memoryerrorsection}) are valid, using boolean fields.
\jsontable{table:memoryerrorvalidationstructure}
errorStatusValid & boolean & Whether the "errorStatus" field of a Memory Error (\ref{section:memoryerrorsection}) is valid.\\
\hline
physicalAddressValid & boolean & Whether the "physicalAddress" field of a Memory Error (\ref{section:memoryerrorsection}) is valid.\\
\hline
physicalAddressMaskValid & boolean & Whether the "physicalAddressMask" field of a Memory Error (\ref{section:memoryerrorsection}) is valid.\\
\hline
nodeValid & boolean & Whether the "node" field of a Memory Error (\ref{section:memoryerrorsection}) is valid.\\
\hline
cardValid & boolean & Whether the "card" field of a Memory Error (\ref{section:memoryerrorsection}) is valid.\\
\hline
moduleValid & boolean & Whether the "module" field of a Memory Error (\ref{section:memoryerrorsection}) is valid.\\
\hline
bankValid & boolean & Whether the "bank.value" field of a Memory Error (\ref{section:memoryerrorsection}) is valid. When the bank is addressed by group/address, refer to \texttt{bankGroupValid} and \texttt{bankAddressValid} instead.\\
\hline
deviceValid & boolean & Whether the "device" field of a Memory Error (\ref{section:memoryerrorsection}) is valid.\\
\hline
rowValid & boolean & Whether the "row" field of a Memory Error (\ref{section:memoryerrorsection}) is valid.\\
\hline
memoryPlatformTargetValid & boolean & Whether the memory platform target of a Memory Error (\ref{section:memoryerrorsection}) is valid.\\
\hline
memoryErrorTypeValid & boolean & Whether the "memoryErrorType" field of a Memory Error (\ref{section:memoryerrorsection}) is valid.\\
\hline
rankNumberValid & boolean & Whether the "rankNumber" field of a Memory Error (\ref{section:memoryerrorsection}) is valid.\\
\hline
cardHandleValid & boolean & Whether the "cardSmbiosHandle" field of a Memory Error (\ref{section:memoryerrorsection}) is valid.\\
\hline
moduleHandleValid & boolean & Whether the "moduleSmbiosHandle" field of a Memory Error (\ref{section:memoryerrorsection}) is valid.\\
\hline
extendedRowBitsValid & boolean & Whether the "extended.rowBit16" and "extended.rowBit17" field of a Memory Error (\ref{section:memoryerrorsection}) is valid.\\
\hline
bankGroupValid & boolean & Whether the "bank.group" field of a Memory Error (\ref{section:memoryerrorsection}) is valid.\\
\hline
bankAddressValid & boolean & Whether the "bank.address" field of a Memory Error (\ref{section:memoryerrorsection}) is valid.\\
\hline
chipIdentificationValid & boolean & Whether the "extended.chipIdentification" field of a Memory Error (\ref{section:memoryerrorsection}) is valid.\\
\jsontableend{Memory Error validation structure field table.}

% Memory error normal bank addressing structure.
\subsection{Memory Error Standard Bank Address Structure}
\label{subsection:memoryerrorstandardbankaddressstructure}
This structure describes a simple bank address for a Memory Error section (\ref{section:memoryerrorsection}). This structure is selected when the \texttt{bankValid} field in the corresponding Memory Error Validation Structure (\ref{subsection:memoryerrorvalidationstructure}) is set to "true".
\jsontable{table:memoryerrorstandardbankaddressstructure}
value & uint64 & The value of the bank address.\\
\jsontableend{Memory Error Standard Bank Address structure field table.}

% Memory error address/group bank addressing structure.
\subsection{Memory Error Address/Group Bank Address Structure}
\label{subsection:memoryerroraddressgroupbankaddressstructure}
This structure describes an address/group bank address for a Memory Error section (\ref{section:memoryerrorsection}). This structure is selected when the \texttt{bankValid} field in the corresponding Memory Error Validation Structure (\ref{subsection:memoryerrorvalidationstructure}) is set to "false".
\jsontable{table:memoryerroraddressgroupbankaddressstructure}
address & uint64 & The address of the bank.\\
\hline
group & uint64 & The group of the bank.\\
\jsontableend{Memory Error Address/Group Bank Address structure field table.}

% Memory error 2 section.
\section{Memory Error 2 Section}
\label{section:memoryerror2section}
This section describes the JSON format for a single Memory Error 2 Section from a CPER record. The GUID used for Memory Error 2 Sections is \texttt{\{ 0x61EC04FC, 0x48E6, 0xD813, \{ 0x25, 0xC9, 0x8D, 0xAA, 0x44, 0x75, 0x0B, 0x12 \}\}}.
\jsontable{table:memoryerror2section}
validationBits & object & A Memory Error 2 Validation structure, as described in Subsection \ref{subsection:memoryerror2validationstructure}.\\
\hline
errorStatus & object & A CPER Generic Error Status structure, as described in Subsection \ref{subsection:genericerrorstatusstructure}.\\
\hline
bank & object & Structure as described in one of Subsection \ref{subsection:memoryerror2standardbankaddressstructure} or Subsection \ref{subsection:memoryerror2addressgroupbankaddressstructure}. Selected structure depends on the \texttt{validationBits.bankValid} field.\\
\hline
memoryErrorType.value & uint64 & The raw value of the memory error type.\\
memoryErrorType.name & string & The human readable name, if available, of the memory error type.\\
\hline
status.value & int & The raw value of the memory error status.\\
status.state & string & The human readable value, if available, of the memory error status (corrected/uncorrected).\\
\hline
physicalAddress & uint64 & The physical address at which the error occurred.\\
\hline
physicalAddressMask & uint64 & Defines the valid address bits in the \texttt{physicalAddress} field.\\
\hline
node & uint64 & Identifies the node containing the memory error, if in a multi-node system.\\
\hline
card & uint64 & The card number of the memory error location.\\
\hline
module & uint64 & The module of the offending memory error location.\\
\hline
device & uint64 & The device number of the memory associated with the error.\\
\hline
row & uint64 & The first 16 bits of the row number of the memory location.\\
\hline
column & uint64 & The column number of the memory error location.\\
\hline
bitPosition & uint64 & The bit position at which the error occurred.\\
\hline
rank & uint64 & The rank number of the error location.\\
\hline
chipID & uint64 & Chip identMrsOp2ifier. Encoded field used to address the die in 3DS packages.\\
\hline
requestorID & uint64 & Hardware address of the device that initiated the errored transaction.\\
\hline
responderID & uint64 & Hardware address of the device that responded to the transaction.\\
\hline
targetID & uint64 & Hardware address of the intended target of the transaction.\\
\hline
cardSmbiosHandle & uint64 & The SMBIOS handle for the memory card's Type 16 Memory Array Structure.\\
\hline
moduleSmbiosHandle & uint64 & The SMBIOS handle for the memory module's Type 17 Memory Device Structure.\\
\jsontableend{Memory Error 2 structure field table.}

% Memory error 2 validation structure.
\subsection{Memory Error 2 Validation Structure}
\label{subsection:memoryerror2validationstructure}
This structure describes whether fields in a single Memory Error 2 (\ref{section:memoryerror2section}) are valid, using boolean fields.
\jsontable{table:memoryerror2validationstructure}
errorStatusValid & boolean & Whether the "errorStatus" field of a Memory Error 2 (\ref{section:memoryerror2section}) is valid.\\
\hline
physicalAddressValid & boolean & Whether the "physicalAddress" field of a Memory Error 2 (\ref{section:memoryerror2section}) is valid.\\
\hline
physicalAddressMaskValid & boolean & Whether the "physicalAddressMask" field of a Memory Error 2 (\ref{section:memoryerror2section}) is valid.\\
\hline
nodeValid & boolean & Whether the "node" field of a Memory Error 2 (\ref{section:memoryerror2section}) is valid.\\
\hline
cardValid & boolean & Whether the "card" field of a Memory Error 2 (\ref{section:memoryerror2section}) is valid.\\
\hline
moduleValid & boolean & Whether the "module" field of a Memory Error 2 (\ref{section:memoryerror2section}) is valid.\\
\hline
bankValid & boolean & Whether the "bank.value" field of a Memory Error 2 (\ref{section:memoryerror2section}) is valid. When the bank is addressed by group/address, refer to \texttt{bankGroupValid} and \texttt{bankAddressValid} instead.\\
\hline
deviceValid & boolean & Whether the "device" field of a Memory Error 2 (\ref{section:memoryerror2section}) is valid.\\
\hline
rowValid & boolean & Whether the "row" field of a Memory Error 2 (\ref{section:memoryerror2section}) is valid.\\
\hline
columnValid & boolean & Whether the "column" field of a Memory Error 2 (\ref{section:memoryerror2section}) is valid.\\
\hline
rankValid & boolean & Whether the "rank" field of a Memory Error 2 (\ref{section:memoryerror2section}) is valid.\\
\hline
bitPositionValid & boolean & Whether the "bitPosition" field of a Memory Error 2 (\ref{section:memoryerror2section}) is valid.\\
\hline
chipIDValid & boolean & Whether the "chipID" field of a Memory Error 2 (\ref{section:memoryerror2section}) is valid.\\
\hline
memoryErrorTypeValid & boolean & Whether the "memoryErrorType" field of a Memory Error 2 (\ref{section:memoryerror2section}) is valid.\\
\hline
statusValid & boolean & Whether the "status" field of a Memory Error 2 (\ref{section:memoryerror2section}) is valid.\\
\hline
requestorIDValid & boolean & Whether the "requestorID" field of a Memory Error 2 (\ref{section:memoryerror2section}) is valid.\\
\hline
responderIDValid & boolean & Whether the "responderID" field of a Memory Error 2 (\ref{section:memoryerror2section}) is valid.\\
\hline
targetIDValid & boolean & Whether the "targetID" field of a Memory Error 2 (\ref{section:memoryerror2section}) is valid.\\
\hline
cardHandleValid & boolean & Whether the "cardSmbiosHandle" field of a Memory Error 2 (\ref{section:memoryerror2section}) is valid.\\
\hline
moduleHandleValid & boolean & Whether the "moduleSmbiosHandle" field of a Memory Error 2 (\ref{section:memoryerror2section}) is valid.\\
\hline
bankGroupValid & boolean & Whether the "bankGroup" field of a Memory Error 2 (\ref{section:memoryerror2section}) is valid.\\
\hline
bankAddressValid & boolean & Whether the "bankAddress" field of a Memory Error 2 (\ref{section:memoryerror2section}) is valid.\\
\jsontableend{Memory Error 2 validation structure field table.}

% Memory error 2 normal bank addressing structure.
\subsection{Memory Error 2 Standard Bank Address Structure}
\label{subsection:memoryerror2standardbankaddressstructure}
This structure describes a simple bank address for a Memory Error 2 section (\ref{section:memoryerror2section}). This structure is selected when the \texttt{bankValid} field in the corresponding Memory Error 2 Validation Structure (\ref{subsection:memoryerror2validationstructure}) is set to "true".
\jsontable{table:memoryerror2standardbankaddressstructure}
value & uint64 & The value of the bank address.\\
\jsontableend{Memory Error 2 Standard Bank Address structure field table.}

% Memory error 2 address/group bank addressing structure.
\subsection{Memory Error 2 Address/Group Bank Address Structure}
\label{subsection:memoryerror2addressgroupbankaddressstructure}
This structure describes an address/group bank address for a Memory Error 2 section (\ref{section:memoryerror2section}). This structure is selected when the \texttt{bankValid} field in the corresponding Memory Error 2 Validation Structure (\ref{subsection:memoryerror2validationstructure}) is set to "false".
\jsontable{table:memoryerror2addressgroupbankaddressstructure}
address & uint64 & The address of the bank.\\
\hline
group & uint64 & The group of the bank.\\
\jsontableend{Memory Error 2 Address/Group Bank Address structure field table.}

% PCIe error section.
\section{PCIe Error Section}
\label{section:pcieerrorsection}
This section describes the JSON format for a single PCIe Error Section from a CPER record. The GUID used for PCIe Error Sections is \texttt{\{ 0xd995e954, 0xbbc1, 0x430f, \{ 0xad, 0x91, 0xb4, 0x4d, 0xcb, 0x3c, 0x6f, 0x35 \}\}}.
\jsontable{table:pcieerrorsection}
validationBits & object & A PCIe Error Validation structure as defined in Subsection \ref{subsection:pcieerrorvalidationstructure}.\\
\hline
portType.value & uint64 & The raw value of the port type for this error.\\
portType.name & string & The human readable name, if available, of the port type for this error.\\
\hline
version.major & int & The major version number for the PCIe specification supported.\\
version.minor & int & The minor version number for the PCIe specification supported.\\
\hline
commandStatus.commandRegister & uint64 & The PCI command register value.\\
commandStatus.statusRegister & uint64 & The PCI status register value.\\
\hline
deviceID & object & A PCIe Device ID structure as defined in Subsection \ref{pciedeviceidstructure}.\\
\hline
deviceSerialNumber & uint64 & The serial number of the device.\\
\hline
bridgeControlStatus.secondaryStatusRegister & uint64 & The bridge secondary status register. \emph{This field is valid for bridges only.}\\
\hline
bridgeControlStatus.controlRegister & uint64 & The bridge control register. \emph{This field is valid for bridges only.}\\
\hline
capabilityStructure.data & string & A base-64 formatted binary dump of the PCIe capability structure for this device. The structure could either be a PCIe 1.1 Capability Structure (36-byte, padded to 60 bytes) or a PCIe 2.0 Capability Structure (60-byte).\\
\hline
aerInfo & object & A PCIe AER Extended Capability structure, as defined in Subsection \ref{subsection:pcieaerecstructure}.\\
\jsontableend{PCIe Error structure field table.}

% PCIe error validation structure.
\subsection{PCIe Error Validation Structure}
\label{subsection:pcieerrorvalidationstructure}
This structure describes which fields within a PCIe Error section (\ref{section:pcieerrorsection}) are valid, using boolean fields.
\jsontable{table:pcieerrorvalidationstructure}
portTypeValid & boolean & Whether the "portType" field within a PCIe Error section (\ref{section:pcieerrorsection}) is valid.\\
\hline
versionValid & boolean & Whether the "version" field within a PCIe Error section (\ref{section:pcieerrorsection}) is valid.\\
\hline
commandStatusValid & boolean & Whether the "commandStatus" field within a PCIe Error section (\ref{section:pcieerrorsection}) is valid.\\
\hline
deviceIDValid & boolean & Whether the "deviceID" field within a PCIe Error section (\ref{section:pcieerrorsection}) is valid.\\
\hline
deviceSerialNumber & boolean & Whether the "deviceSerialNumber" field within a PCIe Error section (\ref{section:pcieerrorsection}) is valid.\\
\hline
bridgeControlStatusValid & boolean & Whether the "bridgeControlStatus" field within a PCIe Error section (\ref{section:pcieerrorsection}) is valid.\\
\hline
capabilityStructureStatusValid & boolean & Whether the "capabilityStructure" field within a PCIe Error section (\ref{section:pcieerrorsection}) is valid.\\
\hline
aerInfoValid & boolean & Whether the "aerInfo" field within a PCIe Error section (\ref{section:pcieerrorsection}) is valid.\\
\hline
\jsontableend{PCIe Error validation structure field table.}

% PCIe Device ID structure.
\subsection{PCIe Device ID Structure}
\label{subsection:pciedeviceidstructure}
This structure describes a PCIe device ID, for use in a PCI Error section (\ref{table:pcieerrorsection}).
\jsontable{table:pciedeviceidstructure}
vendorID & uint64 & The vendor ID of the PCIe device.\\
\hline
deviceID & uint64 & The device ID of the PCIe device.\\
\hline
classCode & uint64 & The class code of the PCIe device.\\
\hline
functionNumber & uint64 & The function number of the PCIe device.\\
\hline
deviceNumber & uint64 & The device number of the PCIe device.\\
\hline
segmentNumber & uint64 & The segment number of the PCIe device.\\
\hline
primaryOrDeviceBusNumber & uint64 & The root port/bridge primary bus number or device bus number of the PCIe device.\\
\hline
secondaryBusNumber & uint64 & The root port/bridge secondary bus number of the PCIe device.\\
\hline
slotNumber & uint64 & The slot number of the PCIe device.\\
\jsontableend{PCIe Device ID structure field table.}

% PCIe Advanced Error Reporting Extended Capability structure.
\subsection{PCIe AER Extended Capability Structure}
\label{subsection:pcieaerecstructure}
This structure describes a PCIe Advanced Error Reporting Extended Capability structure, for use in a PCI Error section (\ref{table:pcieerrorsection}).
\jsontable{table:pcieaerecstructure}
capabilityID & uint64 & The capability ID for this AER structure.\\
\hline
capabilityVersion & uint64 & The capability structure version for this AER structure.\\
\hline
uncorrectableErrorStatusRegister & uint64 & The uncorrectable error status register value.\\
\hline
uncorrectableErrorMaskRegister & uint64 & The uncorrectable error mask register value.\\
\hline
uncorrectableErrorSeverityRegister & uint64 & The uncorrectable error severity register value.\\
\hline
uncorrectableErrorMaskRegister & uint64 & The uncorrectable error mask register value.\\
\hline
aeccReg & uint64 & The AECC register value.\\
\hline
headerLogRegister & string & A base64-encoded binary dump of the header log register.\\
\hline
rootErrorCommand & uint64 & The root error command.\\
\hline
rootErrorStatus & uint64 & The root error status.\\
\hline
errorSourceIDRegister & uint64 & The error source ID register.\\
\hline
correctableErrorSourceIDRegister & uint64 & The correctable error source ID register.\\
\jsontableend{PCIe AER Extended Capability structure field table.}

% PCI/PCI-X Bus error section.
\section{PCI/PCI-X Bus Error Section}
\label{section:pcibuserrorsection}
This section describes the JSON format for a single PCI/PCI-X Bus Error Section from a CPER record. The GUID used for PCI/PCI-X Bus Error Sections is \texttt{\{ 0xc5753963, 0x3b84, 0x4095, \{ 0xbf, 0x78, 0xed, 0xda, 0xd3, 0xf9, 0xc9, 0xdd \}\}}.
\jsontable{table:pcibuserrorsection}
validationBits & object & A PCI/PCI-X Bus Error Validation structure, as described in Subsection \ref{subsection:pcibuserrorvalidationstructure}.\\
\hline
errorStatus & object & A CPER Generic Error Status structure, as described in Subsection \ref{subsection:genericerrorstatusstructure}.\\
\hline
errorType.value & uint64 & The raw value of the error type for this bus error.\\
errorType.name & string & The human readable name, if available, of the error type for this bus error.\\
\hline
busID.busNumber & int & The bus number of this bus ID.\\
busID.segmentNumber & int & The segment number of this bus ID.\\
\hline
busAddress & uint64 & The memory or I/O address on the bus at the time of the error.\\
\hline
busData & uint64 & Data on the bus at the time of the error.\\
\hline
busCommandType & string & The type of command at the time of the error. Either "PCI" or "PCI-X".\\
\hline
busRequestorID & uint64 & The PCI bus requestor ID for the error.\\
\hline
busCompleterID & uint64 & The PCI bus completer ID for the error.\\
\hline
targetID & uint64 & The PCI bus intended target ID for the error.\\
\jsontableend{PCI/PCI-X Bus Error structure field table.}

% PCI/PCI-X Bus error validation structure.
\subsection{PCI/PCI-X Bus Error Validation Structure}
\label{subsection:pcibuserrorvalidationstructure}
This structure describes which fields within a PCI/PCI-X Bus Error section (\ref{section:pcibuserrorsection}) are valid, using boolean fields.
\jsontable{table:pcibuserrorvalidationstructure}
errorStatusValid & boolean & Whether the "errorStatus" field of the PCI/PCI-X Bus Error section (\ref{section:pcibuserrorsection}) is valid.\\
\hline
errorTypeValid & boolean & Whether the "errorType" field of the PCI/PCI-X Bus Error section (\ref{section:pcibuserrorsection}) is valid.\\
\hline
busIDValid & boolean & Whether the "busID" field of the PCI/PCI-X Bus Error section (\ref{section:pcibuserrorsection}) is valid.\\
\hline
busAddressValid & boolean & Whether the "busAddress" field of the PCI/PCI-X Bus Error section (\ref{section:pcibuserrorsection}) is valid.\\
\hline
busDataValid & boolean & Whether the "busData" field of the PCI/PCI-X Bus Error section (\ref{section:pcibuserrorsection}) is valid.\\
\hline
commandValid & boolean & Whether the "busCommandType" field of the PCI/PCI-X Bus Error section (\ref{section:pcibuserrorsection}) is valid.\\
\hline
requestorIDValid & boolean & Whether the "busRequestorID" field of the PCI/PCI-X Bus Error section (\ref{section:pcibuserrorsection}) is valid.\\
\hline
completerIDValid & boolean & Whether the "busCompleterID" field of the PCI/PCI-X Bus Error section (\ref{section:pcibuserrorsection}) is valid.\\
\hline
targetIDValid & boolean & Whether the "targetID" field of the PCI/PCI-X Bus Error section (\ref{section:pcibuserrorsection}) is valid.\\
\jsontableend{PCI/PCI-X Bus Error validation structure field table.}

% PCI/PCI-X Component error section.
\section{PCI/PCI-X Component Error Section}
\label{section:pcicomponenterrorsection}
This section describes the JSON format for a single PCI/PCI-X Component Error Section from a CPER record. The GUID used for PCI/PCI-X Component Error Sections is \texttt{\{ 0xeb5e4685, 0xca66, 0x4769, \{ 0xb6, 0xa2, 0x26, 0x06, 0x8b, 0x00, 0x13, 0x26 \}\}}.
\jsontable{table:pcicomponenterrorsection}
validationBits & object & A PCI/PCI-X Component Error Validation structure, as defined in Subsection \ref{subsection:pcicomponenterrorvalidationstructure}.\\
\hline
idInfo & object & A PCI/PCI-X Component ID structure, as defined in Subsection \ref{subsection:pcicomponentidstructure}.\\
\hline
memoryNumber & uint64 & The number of PCI/PCI-X component memory mapped register address/data pair values are present in this structure.\\
\hline
ioNumber & uint64 & The number of PCI/PCI-X component programmed I/O register address/data pair values are present in this structure.\\
\hline
registerDataPairs & array & An array of PCI/PCI-X Component Register Pair structures, as defined in Subsection \ref{subsection:pcicomponentregisterpairstructure}. The length corresponds to the amounts listed in fields \texttt{memoryNumber} and \texttt{ioNumber}.\\
\jsontableend{PCI/PCI-X Component Error structure field table.}

% PCI/PCI-X Component error validation structure.
\subsection{PCI/PCI-X Component Error Validation Structure}
\label{subsection:pcicomponenterrorvalidationstructure}
This structure describes which fields within a PCI/PCI-X Component Error section (\ref{section:pcicomponenterrorsection}) are valid, using boolean fields.
\jsontable{table:pcicomponenterrorvalidationstructure}
errorStatusValid & boolean & Whether the "errorStatus" field of the PCI/PCI-X Component Error section (\ref{section:pcicomponenterrorsection}) is valid.\\
\hline
idInfoValid & boolean & Whether the "idInfo" field of the PCI/PCI-X Component Error section (\ref{section:pcicomponenterrorsection}) is valid.\\
\hline
memoryNumberValid & boolean & Whether the "memoryNumber" field of the PCI/PCI-X Component Error section (\ref{section:pcicomponenterrorsection}) is valid.\\
\hline
ioNumberValid & boolean & Whether the "ioNumber" field of the PCI/PCI-X Component Error section (\ref{section:pcicomponenterrorsection}) is valid.\\
\hline
registerDataPairsValid & boolean & Whether the "registerDataPairs" field of the PCI/PCI-X Component Error section (\ref{section:pcicomponenterrorsection}) is valid.\\
\jsontableend{PCI/PCI-X Component Error validation structure field table.}

% PCI/PCI-X Component ID structure.
\subsection{PCI/PCI-X Component ID Structure}
\label{subsection:pcicomponentidstructure}
This structure describes the ID of a single PCI/PCI-X component for use in a PCI/PCI-X Component Error section (\ref{section:pcicomponenterrorsection}).
\jsontable{table:pcicomponentidstructure}
vendorID & uint64 & The vendor ID of this PCI/PCI-X component.\\
\hline
deviceID & uint64 & The device ID of this PCI/PCI-X component.\\
\hline
classCode & uint64 & The class code of this PCI/PCI-X component.\\
\hline
functionNumber & uint64 & The function number of this PCI/PCI-X component.\\
\hline
deviceNumber & uint64 & The device number of this PCI/PCI-X component.\\
\hline
busNumber & uint64 & The bus number of this PCI/PCI-X component.\\
\hline
segmentNumber & uint64 & The segment number of this PCI/PCI-X component.\\
\jsontableend{PCI/PCI-X Component ID structure field table.}

% PCI/PCI-X Component Register Pair structure.
\subsection{PCI/PCI-X Component Register Pair Structure}
\label{subsection:pcicomponentregisterpairstructure}
This structure describes a single pair of registers from a PCI/PCI-X component for use in a PCI/PCI-X Component Error section (\ref{section:pcicomponenterrorsection}). The actual "pairs" of address and data aren't necessarily all 16 bytes allocated long, and there is no field to indicate their length, so do not assume that the address is in the first field and the data in the second.
\jsontable{table:pcicomponentregisterpairstructure}
firstHalf & uint64 & The first 8 bytes of the 16 byte register pair structure.\\
\hline
secondHalf & uint64 & The second 8 bytes of the 16 byte register pair structure.\\
\jsontableend{PCI/PCI-X Component Register Pair structure field table.}

% Firmware error section.
\section{Firmware Error Section}
\label{section:firmwareerrorsection}
This section describes the JSON format for a single Firmware Error Section from a CPER record. The GUID used for Firmware Error Sections is \texttt{\{ 0x81212a96, 0x09ed, 0x4996, \{ 0x94, 0x71, 0x8d, 0x72, 0x9c, 0x8e, 0x69, 0xed \}\}}.
\jsontable{table:firmwareerrorsection}
errorRecordType.value & uint64 & The raw value of the type of firmware error record this is.\\
errorRecordType.name & string & The human readable name, if available, of the type of firmware error record this is.\\
\hline
revision & int & The header revision of this record. For the referenced UEFI specification, this value is 2.\\
\hline
recordID & uint64 & Identifier for the referenced firmware error record. When the \texttt{revision} field is greater than 1 (which is expected here), this value will be null.\\
\hline
recordIDGUID & string & GUID of the firmware error record referenced by this section. \textbf{This field is only valid when the \texttt{errorRecordType} field has a value of 2.} Otherwise, this field is ignored.\\
\jsontableend{Firmware Error structure field table.}

% Generic DMAr error section.
\section{Generic DMAr Error Section}
\label{section:dmargenericerrorsection}
This section describes the JSON format for a single Generic DMAr Error Section from a CPER record. The GUID used for Generic DMAr Error Sections is \texttt{\{ 0x5b51fef7, 0xc79d, 0x4434, \{ 0x8f, 0x1b, 0xaa, 0x62, 0xde, 0x3e, 0x2c, 0x64 \}\}}.
\jsontable{table:dmargenericerrorsection}
requesterID & int & The device ID associated with the fault condition.\\
\hline
segmentNumber & int & The segment number associated with the device.\\
\hline
faultReason.value & uint64 & The raw value of the reason for the fault.\\
faultReason.name & string & The human readable name, if available, of the reason for the fault.\\
faultReason.description & string & A human readable description, if available, of the reason for the fault.\\
\hline
accessType.value & uint64 & The raw value of the access type that caused the fault.\\
accessType.name & string & The human readable name, if available, of the access type that caused the fault.\\
\hline
addressType.value & uint64 & The raw value of the addressing type that caused the fault.\\
addressType.name & string & The human readable name, if available, of the addressing type that caused the fault.\\
\hline
architectureType.value & uint64 & The raw value of the DMAr architecture type.\\
architectureType.name & string & The human readable name, if available, of the DMAr architecture type.\\
\hline
deviceAddress & uint64 & The 64-bit device virtual address contained in the faulted DMA request.\\
\jsontableend{Generic DMAr Error structure field table.}

% VT-d DMAr error section.
\section{VT-d DMAr Error Section}
\label{section:vtddmarerrorsection}
This section describes the JSON format for a single VT-d DMAr Error Section from a CPER record. The GUID used for VT-d DMAr Error Sections is \texttt{\{ 0x71761d37, 0x32b2, 0x45cd, \{ 0xa7, 0xd0, 0xb0, 0xfe, 0xdd, 0x93, 0xe8, 0xcf \}\}}.
\jsontable{table:vtddmarerrorsection}
version & int & Version register value as defined in the VT-d specification.\\
\hline
revision & int & Revision field in VT-d specific DMA remapping reporting structure.\\
\hline
oemID & uint64 & OEM ID field in VT-d specific DMA remapping reporting structure.\\
\hline
capabilityRegister & uint64 & Value of VT-d capability register.\\
\hline
extendedCapabilityRegister & uint64 & Value of VT-d extended capability register.\\
\hline
globalCommandRegister & uint64 & Value of VT-d global command register.\\
\hline
globalStatusRegister & uint64 & Value of VT-d global status register.\\
\hline
faultStatusRegister & uint64 & Value of VT-d fault status register.\\
\hline
faultRecord & object & A VT-d DMAR Fault Record structure, as defined in Subsection .\\
\hline
rootEntry & string & A base64-represented binary dump of the root entry table for the associated requester ID.\\
\hline
contextEntry & string & A base64-represented binary dump of the context entry table for the associated requester ID.\\
\hline
pageTableEntry\_Level6 & uint64 & The page table entry for the device virtual address in page level 6.\\
\hline
pageTableEntry\_Level5 & uint64 & The page table entry for the device virtual address in page level 5.\\
\hline
pageTableEntry\_Level4 & uint64 & The page table entry for the device virtual address in page level 4.\\
\hline
pageTableEntry\_Level3 & uint64 & The page table entry for the device virtual address in page level 3.\\
\hline
pageTableEntry\_Level2 & uint64 & The page table entry for the device virtual address in page level 2.\\
\hline
pageTableEntry\_Level1 & uint64 & The page table entry for the device virtual address in page level 1.\\
\jsontableend{VT-d DMAr Error structure field table.}

% VT-d DMAR Fault Record structure.
\subsection{VT-d DMAR Fault Record Structure}
\label{subsection:vtddmarfaultrecordstructure}
This structure describes a fault record, which forms part of a single VT-d DMAr Error section (\ref{section:vtddmarerrorsection}).
\jsontable{table:vtddmarfaultrecordstructure}
faultInformation & uint64 & Fault information field as defined in the VT-d specification.\\
\hline
sourceIdentifier & uint64 & Identifier of the source of the VT-d fault.\\
\hline
privelegeModeRequested & boolean & Whether privelege mode was requested.\\
\hline
executePermissionRequested & boolean & Whether execute permission was requested.\\
\hline
pasidPresent & boolean & Whether the "pasidValue" field contains valid data.\\
\hline
faultReason & uint64 & The reason for the VT-d fault, as defined in the VT-d specification.\\
\hline
pasidValue & uint64 & The PASID associated with the fault.\\
\hline
addressType & uint64 & The addressing type of the fault, as defined by the VT-d specification.\\
\hline
type.value & uint64 & The raw value of the type of VT-d fault record.\\
type.name & string & The human readable name, if available, of the type of VT-d fault record.\\
\jsontableend{VT-d DMAR Fault Record structure field table.}

% IOMMU DMAr error section.
\section{IOMMU DMAr Error Section}
\label{section:iommudmarerrorsection}
This section describes the JSON format for a single IOMMU DMAr Error Section from a CPER record. The GUID used for IOMMU DMAr Error Sections is \texttt{\{ 0x036f84e1, 0x7f37, 0x428c, \{ 0xa7, 0x9e, 0x57, 0x5f, 0xdf, 0xaa, 0x84, 0xec \}\}}.
\jsontable{table:iommudmarerrorsection}
revision & int & The IOMMU specification revision.\\
\hline
controlRegister & uint64 & The IOMMU control register value.\\
\hline
statusRegister & uint64 & The IOMMU status register value.\\
\hline
eventLogEntry & string & A base-64 binary dump of the IOMMU fault-related event log entry, as defined in the IOMMU specification.\\
\hline
deviceTableEntry & string & A base-64 representation of the value from the device table for a given requester ID.\\
\hline
pageTableEntry\_Level6 & uint64 & Page table entry for device virtual address in page level 6.\\
\hline
pageTableEntry\_Level5 & uint64 & Page table entry for device virtual address in page level 5.\\
\hline
pageTableEntry\_Level4 & uint64 & Page table entry for device virtual address in page level 4.\\
\hline
pageTableEntry\_Level3 & uint64 & Page table entry for device virtual address in page level 3.\\
\hline
pageTableEntry\_Level2 & uint64 & Page table entry for device virtual address in page level 2.\\
\hline
pageTableEntry\_Level1 & uint64 & Page table entry for device virtual address in page level 1.\\
\jsontableend{IOMMU DMAr Error structure field table.}

% CCIX PER error section.
\section{CCIX PER Error Section}
\label{section:ccixpererrorsection}
This section describes the JSON format for a single CCIX PER Error Section from a CPER record. The GUID used for CCIX PER Error Sections is \texttt{\{ 0x91335EF6, 0xEBFB, 0x4478, \{0xA6, 0xA6, 0x88, 0xB7, 0x28, 0xCF, 0x75, 0xD7 \}\}}.
\jsontable{table:ccixpererrorsection}
length & uint64 & The length (in bytes) of the entire structure.\\
\hline
validationBits & object & A CCIX PER Validation structure as described in Subsection \ref{subsection:vtddmarfaultrecordstructure}.\\
\hline
ccixSourceID & int & If the agent is an HA, SA, or RA, this indicates the CCIX Agent ID of the reporting component. Otherwise, this is the CCIX Device ID.\\
\hline
ccixPortID & int & The CCIX Port ID that reported this error.\\
\hline
ccixPERLog & string & A base64-represented binary dump of the CCIX PER Log structure, as defined in Section 7.3.2 of the CCIX Base Specification (Rev. 1.0).\\
\jsontableend{CCIX PER Error structure field table.}

% CCIX PER Validation structure.
\subsection{CCIX PER Validation Structure}
\label{subsection:ccixpervalidationstructure}
This structure describes which fields are valid in a CCIX PER Error section (\ref{section:ccixpererrorsection}) using boolean fields.
\jsontable{table:ccixpervalidationstructure}
ccixSourceIDValid & boolean & Whether the "ccixSourceID" field in the CCIX PER Error Section (\ref{section:ccixpererrorsection}) is valid.\\
\hline
ccixPortIDValid & boolean & Whether the "ccixPortID" field in the CCIX PER Error Section (\ref{section:ccixpererrorsection}) is valid.\\
\hline
ccixPERLogValid & boolean & Whether the "ccixPERLog" field in the CCIX PER Error Section (\ref{section:ccixpererrorsection}) is valid.\\
\jsontableend{CCIX PER validation structure field table.}

% CXL Protocol error section.
\section{CXL Protocol Error Section}
\label{section:cxlprotocolerrorsection}
This section describes the JSON format for a single CXL Protocol Error Section from a CPER record. The GUID used for CXL Protocol Error Sections is \texttt{\{ 0x80B9EFB4, 0x52B5, 0x4DE3, \{ 0xA7, 0x77, 0x68, 0x78, 0x4B, 0x77, 0x10, 0x48 \}\}}.
\jsontable{table:cxlprotocolerrorsection}
validationBits & object & A CXL Protocol Validation structure as defined in Subsection \ref{subsection:cxlprotocolvalidationstructure}.\\
\hline
agentType.value & uint64 & The raw value of the detecting CXL agent type.\\
agentType.name & string & The human readable name, if available, of the CXL agent type.\\
\hline
agentAddress & object & One of the structures described in Subsection \ref{subsection:cxlprotocoldeviceagentaddressstructure} or Subsection \ref{subsection:cxlprotocolrcrbaddressstructure}. Included structure is dependent on the \texttt{agentType.value} field.\\
\hline
deviceID & object & A CXL Device ID structure, as defined in Subsection \ref{subsection:cxlprotocoldeviceidstructure}.\\
\hline
deviceSerial & object (\textbf{optional}) & The CXL device serial number. Only included if the detecting device is a CXL device (field \texttt{agentType.value} has value 0).\\
\hline
capabilityStructure & string & A base64-encoded binary dump of the CXL device's PCIe capability structure. This could either be a PCIe 1.1 Capability Structure (36-byte, padded to 60 bytes), or a PCIe 2.0 Capability Structure (60-byte). Only included if the detecting device is a CXL device (field \texttt{agentType.value} has value 0).\\
\hline
dvsecLength & int & Length (in bytes) of the CXL DVSEC structure.\\
\hline
errorLogLength & int & Length (in bytes) of the CXL Error Log structure.\\
\hline
cxlDVSEC & string & A base64-encoded dump of the CXL DVSEC structure. For CXL 1.1 devices, this is a "CXL DVSEC For Flex Bus Devices" structure as defined in the CXL 1.1 specification. For CXL 1.1 host downstream ports, this is the "CXL DVSEC For Flex Bus Port" structure as defined in the CXL 1.1 specification.\\
\hline
cxlErrorLog & string & A base64-encoded dump of the CXL error log. This field contains a copy of "CXL RAS Capability Structure", as defined in the CXL 1.1 specification.\\
\jsontableend{CXL Protocol Error structure field table.}

% CXL Protocol Validation structure.
\subsection{CXL Protocol Validation Structure}
\label{subsection:cxlprotocolvalidationstructure}
This structure describes which fields are valid in a CXL Protocol Error section (\ref{section:cxlprotocolerrorsection}) using boolean fields.
\jsontable{table:cxlprotocolvalidationstructure}
cxlAgentTypeValid & boolean & Whether the "cxlAgentType" field in the CXL Protocol Error section (\ref{section:cxlprotocolerrorsection}) is valid.\\
\hline
cxlAgentAddressValid & boolean & Whether the "cxlAgentAddress" field in the CXL Protocol Error section (\ref{section:cxlprotocolerrorsection}) is valid.\\
\hline
deviceID & boolean & Whether the "deviceID" field in the CXL Protocol Error section (\ref{section:cxlprotocolerrorsection}) is valid.\\
\hline
deviceSerialValid & boolean & Whether the "deviceSerial" field in the CXL Protocol Error section (\ref{section:cxlprotocolerrorsection}) is valid.\\
\hline
capabiltyStructureValid & boolean & Whether the "capabilityStructure" field in the CXL Protocol Error section (\ref{section:cxlprotocolerrorsection}) is valid.\\
\hline
cxlDVSECValid & boolean & Whether the "cxlDVSEC" field in the CXL Protocol Error section (\ref{section:cxlprotocolerrorsection}) is valid.\\
\hline
cxlErrorLogValid & boolean & Whether the "cxlErrorLog" field in the CXL Protocol Error section (\ref{section:cxlprotocolerrorsection}) is valid.\\
\jsontableend{CXL Protocol validation structure field table.}

% CXL Protocol Device Agent Address structure.
\subsection{CXL Protocol Device Agent Address Structure}
\label{subsection:cxlprotocoldeviceagentaddressstructure}
This structure describes the address of a single CXL device agent, for use in a CXL Protocol Error section (\ref{section:cxlprotocolerrorsection}). Included when the \texttt{agentType.value} field has the value "0".
\jsontable{table:cxlprotocoldeviceagentaddressstructure}
functionNumber & uint64 & The function number of the CXL device.\\
\hline
deviceNumber & uint64 & The device number of the CXL device.\\
\hline
busNumber & uint64 & The bus number of the CXL device.\\
\hline
segmentNumber & uint64 & The segment number of the CXL device.\\
\jsontableend{CXL Protocol Device Agent Address structure field table.}

% CXL Protocol RCRB Base Address structure.
\subsection{CXL Protocol RCRB Base Address Structure}
\label{subsection:cxlprotocolrcrbaddressstructure}
This structure describes an RCRB base address, for use in a CXL Protocol Error section (\ref{section:cxlprotocolerrorsection}). Included when the \texttt{agentType.value} field has the value "1".
\jsontable{table:cxlprotocolrcrbaddressstructure}
value & uint64 & The CXL port RCRB base address.\\
\jsontableend{CXL Protocol RCRB Base Address structure field table.}

% CXL Protocol Device ID structure.
\subsection{CXL Protocol Device ID Structure}
\label{subsection:cxlprotocoldeviceidstructure}
This structure describes the ID of a CXL device, for use in a CXL Protocol Error section (\ref{section:cxlprotocolerrorsection}).
\jsontable{table:cxlprotocoldeviceidstructure}
vendorID & uint64 & The vendor ID of the CXL device.\\
\hline
deviceID & uint64 & The device ID of the CXL device.\\
\hline
subsystemVendorID & uint64 & The subsystem vendor ID of the CXL device.\\
\hline
subsystemDeviceID & uint64 & The subsystem device ID of the CXL device.\\
\hline
classCode & uint64 & The class code of the CXL device.\\
\hline
slotNumber & uint64 & The slot number of the CXL device.\\
\jsontableend{CXL Protocol Device ID structure field table.}

% CXL Component error section.
\section{CXL Component Error Section}
\label{section:cxlcomponenterrorsection}
This section describes the JSON format for a single CXL Component Error Section from a CPER record. There are several GUIDs used for CXL Component Error Sections, of which defined are:\\
\begin{itemize}
    \item CXL General Media Error (\texttt{\{ 0xfbcd0a77, 0xc260, 0x417f, \{ 0x85, 0xa9, 0x08, 0x8b, 0x16, 0x21, 0xeb, 0xa6 \}\}})\\
    \item CXL DRAM Event Error (\texttt{\{ 0x601dcbb3, 0x9c06, 0x4eab, \{ 0xb8, 0xaf, 0x4e, 0x9b, 0xfb, 0x5c, 0x96, 0x24 \}\}})\\
    \item CXL Memory Module Error (\texttt{\{ 0xfe927475, 0xdd59, 0x4339, \{ 0xa5, 0x86, 0x79, 0xba, 0xb1, 0x13, 0xb7, 0x74 \}\}})\\
    \item CXL Physical Switch Error (\texttt{\{ 0x77cf9271, 0x9c02, 0x470b, \{ 0x9f, 0xe4, 0xbc, 0x7b, 0x75, 0xf2, 0xda, 0x97 \}\}})\\
    \item CXL Virtual Switch Error (\texttt{\{ 0x40d26425, 0x3396, 0x4c4d, \{ 0xa5, 0xda, 0x3d, 0x47, 0x26, 0x3a, 0xf4, 0x25 \}\}})\\
    \item CXL MLD Port Error (\texttt{\{ 0x8dc44363, 0x0c96, 0x4710, \{ 0xb7, 0xbf, 0x04, 0xbb, 0x99, 0x53, 0x4c, 0x3f \}\}})\\
\end{itemize}
\jsontable{table:cxlcomponenterrorsection}

\jsontableend{CXL Protocol Error structure field table.}

\end{document}